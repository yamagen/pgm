\newif\ifJPN
\JPNtrue
\JPNfalse


\newif\ifTIKZFIG
\TIKZFIGtrue
\TIKZFIGfalse


\ifJPN
  \documentclass[a4paper,xelatex,ja=standard]{bxjsarticle}
\else
  \documentclass[a4paper,xelatex,english,ja=standard]{bxjsarticle}
\fi

%\setCJKmainfont[BoldFont=HaranoAjiMincho-Bold.otf]{HaranoAjiMincho-Regular.otf}
%\setCJKsansfont[BoldFont=HaranoAjiGothic-Bold.otf]{HaranoAjiGothic-Regular.otf}
\setCJKmonofont[BoldFont=HaranoAjiGothic-Bold.otf]{HaranoAjiGothic-Regular.otf}

\ifJPN
\usepackage[backend=biber,style=authoryear]{biblatex-japanese} % BibLaTeX と Biber 
\DeclareLanguageMapping{japanese}{japanese}
\else
\usepackage[backend=biber,style=authoryear]{biblatex} % BibLaTeX と Biber を使用
\fi
\addbibresource{koten.bib} % BibTeX ファイルを指定
\usepackage{enumitem}
\usepackage{tikz}
\usetikzlibrary{arrows.meta, positioning}

\usetikzlibrary{positioning}
\usepackage{version}
\usepackage{amssymb}
\usepackage{amsmath}
\usepackage{booktabs} % このパッケージを追加
\usepackage{graphicx}
\usepackage{standalone}
\usepackage{url}
\usepackage{makeidx}
\usepackage{silence}
\WarningFilter{latexfont}{Font shape}
\WarningFilter{latexfont}{Some font shapes}
\WarningFilter{natbib}{There were undefined citations}
\WarningFilter{natbib}{Package natbib Warning}
\usepackage{metalogo} % \XeLaTeX ロゴのため
\usepackage{fancyvrb}
\usepackage{color}
\usepackage{amsmath}
\usepackage{listings} % シンプルで効果的なコードリスト表示
\lstset{
  language=Python,         % Pythonのシンタックスハイライト
  frame=t,            % 全体を枠線で囲む
  basicstyle=\ttfamily\footnotesize, % コードのフォントe定
  numbers=left,            % 行番号を左に表示
  numberstyle=\tiny,       % 行番号のフォントを小さくする
  stepnumber=1,            % 行番号を1行ごとに表示
  tabsize=4,               % タブのサイズを指定
  breaklines=true,         % 長い行を自動的に折り返す
  showstringspaces=false,  % 文字列の空白を表示しない
  keywordstyle=\color{blue}, % キーワードに色を付ける
  commentstyle=\color{black}, % コメントに色を付ける
  stringstyle=\color{red}, % 文字列に色を付ける
}
\usepackage{layout}
%\usepackage{natbib}
\usepackage{support-caption}
\usepackage[format=hang,labelsep=colon,margin=10pt,sc,normalsize]{caption}
%\captionsetup[table]{skip=0pt}
%\captionsetup[figure]{skip=10pt}
%\bibpunct[:\,]{(}{)}{,}{a}{}

\usepackage{hyperref}
\usepackage{url}
\usepackage{makeidx}



\makeindex

\ifJPN
%\captionsetup[table]{name=表}
%\captionsetup[figure]{name=図}
\renewcommand{\refname}{文献}
\renewcommand{\indexname}{索引}
\else
%\captionsetup[table]{name=Table~}
%\captionsetup[figure]{name=Figure~}
\renewcommand{\refname}{Reference}
\renewcommand{\indexname}{Index}
\fi

\if0
\newcounter{marginparcnt}[chapter]
\newcommand{\theMarginparcnt}{$\dagger$\arabic{marginparcnt}}
\newcommand{\Marginpar}[2][-7pt]{%
  \stepcounter{marginparcnt}%
  \textcolor{red}{\textsuperscript{\theMarginparcnt}}%
  \protect\marginpar{\vskip#1\footnotesize\color{blue}%
    \textsuperscript{\theMarginparcnt}
    {#2}\par}}
  \fi

\ifJPN
\title{プロセス文法モデル\\即時文法と調整文法}
\author{
  山元 啓史\\東京科学大学
%\and
%  \Large\textbf{ホドシチェク ボル}\\\Large\textbf{大阪大学}
}
\else
\title{Process Grammar Model\\Immediate Grammar and Adjustive Grammar}
\author{
Hilofumi Yamamoto\\Institute of Science Tokyo
%\and
%Bor Hodošček\\The University of Osaka
} 
\fi
\date{\today}

\begin{document}
\maketitle



\ifJPN
\section{はじめに}
\else
\section{Introduction}
\fi

\ifJPN
即時文法は、発話が直感的に選ばれ、迅速に実現される状況に対応する文法である。
調整文法は、適切性の判断や調整が加えられた発話に用いられる文法である。
それら、すぐに話さなければならない状況と、じっくり考えて調整して話すという二つの極を持つ言語使用を記述するための枠組みとして、プロセス文法モデルを提案する。
即時文法は何でもよいからすぐに話しさえすればよいのではなく、これには厳格なルールがある。
調整文法は適切な言葉を選び、適切な文法を使うことが重要であることには変わりないが、調整の程度(あるいは推敲の程度)により、いくつもの表現方法が存在し、調整過剰というやりすぎ状態が見られる。
表現の時間軸を考慮したこのモデルは、すべてこれまでの文法研究とは異なるものではなく、これまでの文法研究をさらに発展させるためのフレームワークである。
\else
Immediate grammar is a grammar that corresponds to the situation where utterances are intuitively selected and promptly realized.
Adjustive grammar is a grammar used for utterances with appropriate judgment and adjustive.
As a framework for describing language use with two extremes, one that must be spoken immediately and one that is spoken after careful consideration and adjustive, we propose a process grammar model.
Immediate grammar is not just about speaking anything right away, but there are strict rules for this.
Adjustive grammar is important to choose the right words and use the right grammar, but depending on the degree of adjustive (or the degree of revision), there are several ways of expression, and there is a state of over-adjustment.
This model, which considers the time axis of expression, is not something completely different from all previous grammar studies, but a framework for further developing previous grammar studies.
\fi

\ifJPN
このモデルの特徴は、時間軸を考慮した言語における文法の動的特性を記述することにある。
\else
The feature of this model is to describe the dynamic characteristics of grammar in language considering the time axis.
\fi
\ifJPN
言語は物理リソース(脳の認知プロセス、発声、記号操作など)を介して運用されるため、数理的に記述することが可能である。
しかし、言語は単なる物理的システムではなく、物質を記述する各々のパーツが物理学でいわれる物理量とは性質が異なるため、数理的な記述には困難が伴う。
物理システム(熱力学、電磁気学など)は通常、決定論的な法則に従う。
一方、言語は「即時文法」と「調整文法」のように、非決定論的な要素を含む。
特に、言語には文脈依存性、意味構造、認知プロセスの影響が含まれ、単純な数理モデルでは完全に説明できない。
それゆえ、言語の数理的記述は、物理的リソースを基盤としつつも、動的な変化や意思決定プロセスを考慮する必要がある。
たとえば、発話のリアルタイム生成(即時文法) は、決定論的な法則で完全に予測することが困難であるため、相対的に捉え、二項対立のように常に比較すべき対を示す。
たとえば、意味の曖昧性(例:「娘がいる」はmy daughter なのか、a girl なのか)や、文脈依存性(例:「彼女は彼に会った」は、she met him なのか、he met her なのか)などは、数理的には解決が難しい。
これらの問題点について放置するわけにもいかないし、統計的確率論として扱うわけにもいかない。
しかしながら、時間軸を考慮した言語の数理モデルは、未解決であったこれらの問題に対して、新たなアプローチを提供する可能性がある。
たとえば、即時文法であるならば、発話のリアルタイム生成を考慮し、my daughter/a girl という問題は、発話の前後関係によって解決される。
また、調整文法であるならば、文脈依存性を考慮し、she met him/he met her という曖昧性の問題は、調整の過程によって語句を追加すること解決される。
    \else
    Language is operated through physical resources (cognitive processes of the brain, speech, symbol manipulation, etc.), making it possible to describe it mathematically.
However, language is not just a physical system, and each part that describes matter has different properties from the physical quantities described in physics, making mathematical descriptions difficult.
Physical systems (thermodynamics, electromagnetism, etc.) usually follow deterministic laws.
On the other hand, language contains non-deterministic elements such as ``immediate grammar'' and ``adjustment grammar.''
In particular, language includes context dependency, semantic structure, and cognitive process influences, which cannot be fully explained by simple mathematical models.
Therefore, a mathematical description of language must consider dynamic changes and decision-making processes while being based on physical resources.
For example, real-time generation of speech (immediate grammar) is difficult to predict completely with deterministic laws, so it should be relatively understood and always show pairs to be compared like a binary opposition.
For example, semantic ambiguity (e.g., ``She has a daughter'' is it my daughter or a girl?) and context dependency (e.g., ``She met him'' is it she met him or he met her?) are difficult to resolve mathematically.
These problems cannot be ignored, nor can they be simply addressed by statistical probability theory.
However, a mathematical model of language that considers the time axis may offer a new approach to these unresolved issues.
For example, if it is immediate grammar, considering the real-time generation of speech, the problem of my daughter or a girl is resolved by the relationship before and after the speech.
Also, if it falls under adjustment grammar, considering context dependency, the problem of ambiguity between she met him and he met her is resolved by adding words during the adjustment process.
    \fi

\ifJPN
\section{理論的背景}
\else
\section{Theoretical Background}
\fi

\ifJPN
\ref{qa:20250405a}も併せて参照してください。
\else
Please refer to \ref{qa:20250405a} as well.
\fi

\ifJPN
  \subsection{二重過程理論との関連}
\else
  \subsection{Relation to Dual Process Theory}
\fi

\begin{figure}[htb]\centering\small
\includegraphics[width=0.45\textwidth]{./figures/fastslow01.pdf} 
\ifJPN
  \caption{二重過程理論 \href{https://thedecisionlab.com/reference-guide/philosophy/system-1-and-system-2-thinking}{``System 1 and System 2 Thinking''}より}\label{fig:fastslow01-j}
\else
  \caption{Dual processing theory based on the presentation from \href{https://thedecisionlab.com/reference-guide/philosophy/system-1-and-system-2-thinking}{``System 1 and System 2 Thinking''}}\label{fig:fastslow01}
\fi
\end{figure}

\begin{table}[htb]\centering\small
\ifJPN
\caption{システム1とシステム2との特徴の違い}
\else
\caption{Differences in characteristics between System 1 and System 2}
\fi
\label{tab:system1-system2}
\ifJPN
\toprule
\begin{tabular}{@{}ll@{}}
\textbf{System 1} / \textbf{発話例(即時的・直感的)} & \textbf{System 2} / \textbf{発話例(慎重・調整的)} \\
\midrule
速い: 「うわっ、熱っ!」 & 遅い:「やけどする温度ですから気をつけてください」 \\
潜在意識: 「なんか嫌な感じがする…」 & 意識的:「この状況はリスク要因が多いですね」 \\
自動的: 「あ、どうぞどうぞ!」 & 努力的:「お先にどうぞ、と言うべきかな…?」 \\
経験則: 「いつもこうだから同じでしょ」 & 分析的:「過去5回のデータから見ても今回は違いそうだ」 \\
連想的: 「この匂い、なんか懐かしい」 & ルールベース:「これは〇〇理論のパターンに当てはまります」 \\
暗黙的: 「うん、まぁそんな感じ」 & 明示的:「つまりAがBに影響してCという結果になります」 \\
感情的: 「もう無理!やだ!」 & 論理的:「その方法には問題点が3つあります」 \\
直感的: 「なんか違う気がする」 & 合理的:「コストとリスクを考えるとこちらのほうが妥当です」 \\
誤りやすい: 「たぶん大丈夫でしょ!」 & 信頼性が高い:「複数のソースを確認した上で判断しました」 \\
\end{tabular}
\else
\begin{tabular}{@{}lp{50mm}lp{60mm}@{}}
\toprule
\textbf{System 1} & \textbf{Example utterance } & \textbf{System 2} & \textbf{Example utterance} \\
& \textbf{immediate} & & \textbf{deliberate} \\
\midrule
Fast& ``Whoa, that's hot!'' & Slow& ``Be careful, it's hot enough to burn you.'' \\
Subconscious& ``I have a bad feeling about this...'' & Conscious& ``There are several risk Factors here.'' \\
Automatic& ``Oh, go ahead!'' & Effortful& ``Should I say 'please go ahead'...?'' \\
Heuristic& ``It's always like this, so it'll be the same this time.'' & Analytical& ``Looking at the past five cases, this one might be different.'' \\
Associative& ``This smell... reminds me of something.'' & Rule-based& ``This fits the typical pattern according to theory X.'' \\
Implicit& ``Yeah, kind of like that.'' & Explicit& ``A leads to B, and that causes C.'' \\
Emotional& ``I can't take this anymore!'' & Logical& ``There are three problems with that approach.'' \\
Intuitive& ``Something just feels off.'' & Rational& ``Considering the costs and risks, this option is more reasonable.'' \\
Error-prone& ``It'll probably be fine!'' & Reliable& ``I've verified it with multiple sources.'' \\
\end{tabular}
\fi
\end{table}

\ifJPN
プロセス文法モデルは、二重過程理論に言われる「システム1(速い思考)とシステム2(遅い思考)」の枠組みに類似する。しかし、本モデルは単なる応用ではなく、即時性と調整性に焦点を当てた独自の視点を持つ\autocite{Evans2008, Kahneman2011-KAHTFA-2, squire2009memory}。
\else
The Process Grammar Model is similar to the framework of ``System 1 (fast thinking) and System 2 (slow thinking)'' refered as the Dual Process Theory. However, this model has a unique perspective focusing on immediacy and adjustability, rather than being a mere application.\autocite{Evans2008, Kahneman2011-KAHTFA-2, squire2009memory}
\fi

\ifJPN
  \subsection{「話しことば」「書きことば」という言い方を改める}
\else
  \subsection{Limitation of ``spoken language'' and ``written language''}
\fi

\ifJPN
従来の言語学では、「話しことばは即時性が高く、書きことばは推敲可能である」とされてきた。しかし、SNSやライブ配信では「書きながら話す」状況が増えており、単純な二分法では説明できない。
こうした現象を整理し、表面上の行為を言語形式として呼ぶことを止め、言語使用のメカニズムを説明するための枠組みとしてプロセス文法モデルという名称を提案するものである。
\else
In traditional linguistics, it has been said that ``spoken language has high immediacy and written language is revisable.'' However, in SNS and live streaming, the situation of ``speaking while writing'' is increasing, and it cannot be explained by a simple binary opposition.
This study proposes the name ``Process Grammar Model'' as a framework to explain the mechanism of language use, rather than calling surface acts as language forms, to organize such phenomena.
\fi

\ifJPN
  \subsection{節連接と継ぎ足し構文}
\else
  \subsection{Clause chaining and additive constructions}
\fi

\ifJPN
「即時文法」はこれまでに研究されてきた言語現象としては節連接が最も近い。
節連接は、文法的な連続性を示す手法として広く研究されてきた。節連接は、文の構造や意味を維持しながら、複数の節を連結することで、情報の連続性や関係性を表現する。節連接は、即時文法と調整文法の連続体を説明する上で有用な手法である。
\else
``Immediate grammar'' is most closely related to the linguistic phenomenon of clause chaining that has been studied to date. Clause chaining has been widely studied as a method to indicate grammatical continuity. By linking multiple clauses while maintaining the structure and meaning of sentences, clause chaining expresses the continuity and relationships of information. Clause chaining is a useful method for explaining the continuum of immediate and adjustment grammars.
\fi

\ifJPN
%  \textcite{sakakura1975aj,komatu2003a,kondo2005aj}は、節連接の文法的特徴と意味的な関係性について分析を行っている。
  \textcite{sakakura1975aj}は、和文の構成を「つけ足しつけ足ししながら」と表現し、また\textcite{komatu2003a}は「つぎつぎと継ぎたして構成される」と述べて、和歌や和文における節連鎖の特性を指摘している。このような表現は、文の構成が逐次的で柔軟であることを強調しており、各句節間の相互関係が必ずしも緊密でないことを示唆している。
  \textcite{kondo2005aj}も「継ぎたされるわけであるので、結果としてできた複文においては、どれが主節でどれが従属節かというようなことは定義しにくい」と述べ、節の階層構造が曖昧であることを指摘している。
  これらの考察は、言語使用が構文としての拘束ルールに従うよりも、心理学的に行為やイベントが想起される順に発話されることに自然に基づいていることを示唆しており、言語の発話は文法的な調整よりも、話者の思考の流れや状況に応じた柔軟な適応によって成り立つことが明らかである。
  したがって、このような即時的な発話プロセスは、言語の発達やコミュニケーションにおいて普遍的に見られる特徴であると考えられる。
\else
%  \textcite{sakakura1975ae,komatu2003ae,kondo2005be} have analyzed the grammatical characteristics and semantic relationships of clause chaining.
  \textcite{sakakura1975ae} expressed the construction of Japanese text as ``adding one after another,'' and \textcite{komatu2003ae} stated that it is ``constructed by adding one after another,'' pointing out the characteristics of clause chaining in Japanese poetry and prose. Such expressions emphasize that the construction of sentences is sequential and flexible, suggesting that the relationships between each phrase are not necessarily close. \textcite{kondo2005be} also stated that ``since it is added, it is difficult to define which is the main clause and which is the subordinate clause in the resulting compound sentence,'' indicating that the hierarchical structure of clauses is ambiguous. These considerations suggest that language use is based on the natural order in which acts and events are recalled, rather than following syntactic binding rules. They reveal that language utterances are based on flexible adaptations to the flow of the speaker's thoughts and the situation, rather than grammatical adjustments. Therefore, such immediate speech processes are considered to be universally observed features in language development and communication.
\fi

\ifJPN
  \subsection{節連接と句の連結方法}
\else
  \subsection{Clause chaining and linking methods of phrases}
\fi

\ifJPN
\textcite{10.3389/fpsyg.2019.03008}は、日本語の物語における節連接の使用について研究しており、60人の子供(3~7歳)と10人の大人に絵本のストーリーテリングや短編ビデオを見た後で物語を再話した。
この研究では、句の連結方法やその終了のタイミング、意味的な関係(同時性、因果関係、時間的順序など)について詳しく調べられ、句を終了させるタイミングは、登場人物が変わる時や物語の重要な単位の終わりに関連していることが確認された。
また、子供たちと大人の物語を比較した結果、句の連結に影響を与える要因は年齢に関係なく存在することが示された。

子供たちの物語における反応は、即時文法に近い形で自然に現れていると考えられ、物語を語る際、子供たちは短い単位で反応を繰り返すことが多く、句の連鎖が必ずしも論理的に整然としていない場合があると指摘されている。
これは、即時文法の特徴である、即座に文を構成していくことを反映していると考えられる。
子供たちは、言語的な調整が少なく、直感的に次の句に繋げるため、調整的な文法規則が発達していない段階で即座に反応することが主導的で、この即時的な反応は、言語発達の過程において重要な役割を果たし、複雑な構文や文法的な調整を学ぶための土台となっている。
子供たちが文を作り出す際の柔軟さや流動性は、即時文法的な特徴と関連していると考えられる。
\else
  \textcite{10.3389/fpsyg.2019.03008} studied the use of clause chaining in Japanese narratives, where 60 children (aged 3-7) and 10 adults retold stories after viewing picture book storytelling and short videos. This study examined the methods of linking phrases, the timing of their completion, and semantic relationships (simultaneity, causality, temporal order) in detail. The timing of phrase completion was confirmed to be related to changes in characters and the end of important story units. Furthermore, the results of comparing children's and adults' stories showed that factors influencing the linking of phrases exist regardless of age.  

  The responses in children's stories are thought to naturally appear in a form close to immediate grammar. When telling stories, children often repeat responses in short units, and it is noted that the chaining of phrases is not always logically coherent. This is considered to reflect the characteristic of immediate grammar, where sentences are constructed immediately. Children's immediate responses are dominant at a stage where linguistic adjustments are minimal, and they intuitively connect to the next phrase. This immediate response plays an important role in the process of language development and serves as a foundation for learning complex syntax and grammatical adjustments. The flexibility and fluidity that children exhibit when creating sentences are considered to be related to the characteristics of immediate grammar.
\fi

%\ifJPN
%  \subsection{関連研究}
%\else
%  \subsection{Related Research}
%\fi

\ifJPN
  \section{即時文法と調整文法}
\else
  \section{Immediate Grammar and Adjustive Grammar}
\fi

\ifJPN
  \begin{table}[htb]\centering\small
  \label{tab:immediate-vs-adjustive-j}
  \caption{即時文法 vs 調整文法の対比}
  \begin{tabular}[c]{lll}\noalign{\hrule height .8pt}
  カテゴリ   & 即時文法  & 調整文法\\ \hline
  名詞止め   & ちょうど今、カレーうどん、できたところ。&カレーうどんがちょうど今、できたところです。\\
             & いわば、音楽でいうと楽譜。& これは、音楽における楽譜に相当します。\\
             & 午後は、雨。& 午後は、雨です。\\
             & お昼は、マクドナルドで。& お昼は、マクドナルドで食べます。\\
             & 図書館で、勉強。& 今日は、図書館で勉強します。\\
  重要語が先 & どうですか、私の説明は?& 私の説明は、適切に伝わっていますでしょうか?\\
  副詞のみ& 確かに。&確かに、その指摘は妥当です。\\
  指示詞のみ& これが、こうなる。&	この要素が、このように変化します。\\
    \end{tabular}
\end{table}
\else
\begin{table}[htb]\centering\small
  \caption{Comparison of Immediate Grammar and Adjustive Grammar}
  \label{tab:immediate-vs-adjustive}
  \begin{tabular}[c]{lll}\noalign{\hrule height .8pt}
    & Immediate Grammar & Adjustive Grammar \\ \hline
    Nouns, noun phrases only&
    Just now, curry udon, done.	&The curry udon has just been completed.\\
                                &
    So to speak, like music notation.	&This corresponds to music notation.\\
    Omitting ``wa'' particle&
    How about my explanation?	&Is my explanation being conveyed properly?\\
    Adverbs only&
    Certainly.	&Certainly, that point is valid.\\
    Demonstratives only&
    This becomes like this.	&This element changes like this.\\
  \end{tabular}
\end{table}
\fi

\ifJPN
即時文法は短く、リアルタイム性が高い。最小限の単位で成立する。
従来は文として見做されないとされてきたが、このような形式は、SNSやライブ配信などでより一層、顕著に、そして広く使われている。
調整文法は文の構造が明確で、推敲が可能な文法である。

即時文法の実在性を確認するには、一般的に文と呼ばれるものを短くし、目の前の人に対して、短く発話できるとき、その形式を捉え、「名詞止め」「副詞のみ」などの命名で分類すると、その特徴的なルールが見えてくることだろう。
ここでは、即時文法と調整文法の特徴を端的に示すために、両者の対比を表\ref{tab:immediate-vs-adjustive-j}に示す。
\else
Immediate grammar is short and has high real-time characteristics. It is established in the smallest unit.
Traditionally, it has not been considered as a sentence, but such forms are more prominent and widely used in SNS and live streaming.
Adjustive grammar has a clear sentence structure and is a grammar that can be revised.

To confirm the existence of immediate grammar, it is generally necessary to shorten what is commonly called a sentence and capture the form when it can be spoken briefly to the person in front of you, classifying it with names such as ``noun stop'' and ``adverbs only'' to reveal its characteristic rules.
Here, to succinctly show the characteristics of immediate grammar and adjustive grammar, the comparison of the two is shown in Table \ref{tab:immediate-vs-adjustive}.
\fi

\ifJPN
  \subsection{即時文法(Immediate Grammar)}
\else
  \subsection{Immediate Grammar}
\fi

\ifJPN
即時文法は、リアルタイムで言語を生成する動的プロセスとして形式化される。このモデルは、入力(状況や文脈)に基づいて即時的に適切な出力を決定するシステムとして記述される。
\else
Immediate grammar is formalized as a dynamic process of generating language in real-time. This model is described as a system that determines appropriate output immediately based on input (situation and context).
\fi

\ifJPN
  \begin{description}
    \item[特徴:] 瞬時の発話、省略が多い、文脈依存、リアルタイム処理
    \item[例:] 「どうですか、味?」「あ、見て!あれ」「あぶない」「普通が一番」
    \item[適用場面:] 日常会話、あいづち、緊急時の発話
  \end{description}
\else
  \begin{description}
    \item[Characteristics:] Instantaneous speech, frequent omissions, context-dependent, real-time processing
    \item[Examples:] ``How's the taste?,'' ``Look!,'' ``Watch out!,'' ``Ordinary is the best''
    \item[Application:] Everyday conversation, interjections, emergency speech
  \end{description}
\fi


\ifJPN
\subsubsection{基本モデル}
\else
\subsubsection{Basic Model}

\ifJPN
プロセス文法における発話生成の関数型記法で説明すると、即時文法は以下の関数で定義される。
\else
In the Process Grammar Model, immediate grammar is defined by the following function in functional notation.
\fi

\[
  f_{\text{IG}} : (C, M, T) \to O
\]

\ifJPN
ただし、\( C \) はコンテキスト(状況、文脈)を表し、話し手と聞き手の関係や発話の場面、視覚的情報などが含まれる。\( M \) は言語的知識(語彙や文法構造)を指し、母語話者が持つ即時的な文法的規則を含む。\( T \) は時間的制約(リアルタイム処理)を表し、発話が遅延なく生成される条件を示す。\( O \) は出力(生成された言語表現)を意味し、実際の発話や書きことばが含まれる。即時文法は、入力 \((C, M, T)\) に基づき、リアルタイムで最適な言語表現 \( O \) を決定するプロセスである。
\else
Here, \( C \) represents the context (situation, background), which includes the relationship between the speaker and listener, the setting of the utterance, and visual information. \( M \) refers to linguistic knowledge (vocabulary and grammatical structures), including the immediate grammatical rules possessed by native speakers. \( T \) represents temporal constraints (real-time processing), indicating conditions under which utterances are generated without delay. \( O \) signifies output (generated language expression), encompassing actual speech or written language. Immediate grammar is a process that determines the optimal language expression \( O \) in real-time based on input \((C, M, T)\).
\fi

\ifJPN
\subsubsection{即時性の条件}
\else
  \subsubsection{Condition of Immediacy}
\fi

\ifJPN
即時文法の中核は「即時性」であり、次の条件で定義される。
\else
The core of immediate grammar is ``immediacy,'' defined by the following condition.
\fi

\[
  t_{pr} \leq t_{th}
\]

\ifJPN
ただし、
\( t_{pr} \)は言語生成の処理時間、
\( t_{th} \)は即時性を満たすための閾値(一般に短時間)を表す。
\( t_{pr} \) が \( t_{th} \) を超える場合、そのプロセスは即時文法に属さない。
\else
  Here, \( t_{pr} \) represents the processing time for language generation, and \( t_{th} \) is the threshold for immediacy (generally a short duration). If \( t_{pr} \) exceeds \( t_{th} \), the process does not belong to immediate grammar.
\fi  

\ifJPN
\subsubsection{動的適応性}
\else
\subsubsection{Dynamic Adaptability}
\fi

\ifJPN
即時文法は、入力の変化に応じて適応的に出力を変更する必要がある。この動的適応性は、次の最適化問題として表される。
\else
Immediate grammar must adaptively change output in response to changes in input. This dynamic adaptability is expressed as the following optimization problem.
\fi

\[
\arg\max_{O} U(O \mid C, M, T)
\]

\ifJPN
ただし、\( U(O \mid C, M, T) \) は出力 \( O \) の適切性を示すユーティリティ関数である。目標は、文脈 \( C \)、言語知識 \( M \)、時間的制約 \( T \) に基づき、最適な出力 \( O \) を生成することである。
\else
Here, \( U(O \mid C, M, T) \) is a utility function indicating the appropriateness of output \( O \). The goal is to generate the optimal output \( O \) based on context \( C \), linguistic knowledge \( M \), and temporal constraints \( T \).
\fi

\ifJPN
\subsubsection{連続性の表現}
\else
\subsubsection{Representation of Continuity}
\fi

\ifJPN
即時文法は離散的な生成ではなく、連続的なプロセスとして捉えられる。この連続性は、状況がわずかに変化したときに出力も滑らかに変化するという特性として表される。この連続性は、動的変化への即応として機能し、環境や文脈の変化に対してリアルタイムで適応することを意味する。
\else
Immediate grammar is not viewed as discrete generation but as a continuous process. This continuity is expressed as the characteristic that output changes smoothly when the situation changes slightly. This continuity functions as an immediate response to dynamic changes, meaning it adapts in real-time to changes in the environment or context.
\fi

\[
\frac{\partial O}{\partial C}, \quad \frac{\partial O}{\partial T} \neq 0
\]

\ifJPN
ここで使われている小文字のデルタ(\( \partial \))は、偏微分と呼ばれるもので、複数の要素が影響を与える中で、特定の要素だけをわずかに変化させたときに出力がどのように変化するかを示す。この分数の形は、出力 \( O \) がコンテキスト \( C \) や時間 \( T \) の微小な変化に対してどれくらい敏感に反応するか、つまり「変化量の比率」を示している。
\else
Here, the lowercase delta (\( \partial \)) represents partial derivatives, which indicate how output changes when a specific element is slightly varied while multiple elements are influencing it. The fractional form shows how sensitive output \( O \) is to small changes in context \( C \) and time \( T \), indicating the ``rate of change''.
\fi

\ifJPN
重要なのは、これらの偏微分が0ではない(\( \neq 0 \))という点である。これは、コンテキスト \( C \) や時間 \( T \) にわずかな変化が生じた場合でも、出力 \( O \) が無反応ではなく、必ず何らかの変化を示すことを意味する。もし偏微分が0であれば、どれほど状況が変わっても出力は変化しないことになるが、即時文法は常に外部の変化に敏感に適応する。この特性は、たとえば話し相手の表情や声のトーンがほんの少し変わっただけで、無意識に反応が変わる現象に似ている。
\else
The important point is that these partial derivatives are not zero (\( \neq 0 \)). This means that even if there are slight changes in context \( C \) or time \( T \), output \( O \) will not be unresponsive and will always show some change. If the partial derivatives were zero, it would mean that no matter how much the situation changes, the output would not change. However, immediate grammar always adapts sensitively to external changes. This characteristic is similar to the phenomenon where even a slight change in the facial expression or tone of voice of the conversation partner unconsciously alters the response.
\fi

\ifJPN
即時文法は、状況の変化に対して意識的な調整を必要としないほど迅速に反応するプロセスである。確かにプロセス文法モデルは、二重過程理論における人間の行動理解から発想を得ているが、その核心は即時性にあり、二重過程理論で示される無意識的または自動的なプロセスとの共通点は、即時的な反応の結果として生じるにすぎない。
\else
Immediate grammar is a process that responds so quickly to changes in the situation that it does not require conscious adjustments. While the Process Grammar Model is inspired by human behavioral understanding in dual process theory, its core lies in immediacy, and the commonalities with unconscious or automatic processes shown in dual process theory are merely results of immediate responses.
\fi

\ifJPN
一方、調整文法における連続性は、即応性ではなく、思考(熟考)レベルでの連続性として現れる。これは、意識的な判断や選択の中で、時間をかけて発話内容や形式が調整されていく過程を指す。したがって、即時文法と調整文法は異なる種類の連続性を持ちながら、言語使用において相互に補完的な役割を果たしている。
\else
On the other hand, continuity in adjustive grammar appears as continuity at the level of thought (deliberation) rather than immediate responsiveness. This refers to the process where utterance content and form are adjusted over time through conscious judgment and selection. Therefore, while immediate grammar and adjustive grammar have different types of continuity, they play complementary roles in language use.
\fi

\ifJPN
\subsubsection{即時文法の全体式}
\else
\subsubsection{Overall Expression of Immediate Grammar}
\fi

\ifJPN
以上を統合し、即時文法を次の形式で表現する。
\else
We can express immediate grammar in the following form by integrating the above elements.
\fi

\[
O = f_{\text{IG}}(C, M, T), \quad \text{s.t.} \quad t_{\text{pr}} \leq t_{\text{th}}
\]

\ifJPN
ただし、即時文法の出力 \( O \) は、状況 \( C \)、言語的知識 \( M \)、時間的制約 \( T \) に基づく関数 \( f_{\text{IG}} \) によって決定されるが、その処理時間 \( t_{\text{pr}} \) は、許容される時間閾値 \( t_{\text{th}} \) を超えてはならない。  
ここで、リアルタイムの適応性を最大化しながら、制約条件を満たす出力 \( O \) を生成することが即時文法の本質である。
\else
  Here, the output \( O \) of immediate grammar is determined by the function \( f_{\text{IG}} \) based on the situation \( C \), linguistic knowledge \( M \), and temporal constraints \( T \). However, its processing time \( t_{\text{pr}} \) must not exceed the allowable time threshold \( t_{\text{th}} \). The essence of immediate grammar is to generate output \( O \) that satisfies the constraints while maximizing real-time adaptability.
\fi

%\input{./immediateGrammar.tex}

\ifJPN
  \subsection{調整文法(Adjustive Grammar)}
\else
  \subsection{Adjustive Grammar}
\fi

\ifJPN
  \begin{description}
    \item[特徴:] 慎重な選択、文法的に整った構造、推敲を経る
    \item[例:] 「この研究の結果から考察すると...」「誠にありがとうございます」
    \item[適用場面:] 公式スピーチ、論文、公文書
  \end{description}
\else
  \begin{description}
    \item[Characteristics:] Careful selection, grammatically correct structure, revision
    \item[Examples:] ``Based on the results of this study...,'' ``Thank you very much''
    \item[Application:] Formal speeches, papers, official documents
  \end{description}
\fi

%\input{adjustiveGrammar.tex}

\ifJPN
調整文法は、即時文法とは異なり、文法形式に対する意識的な調整が行われる。
最たる例は、形式文法である。
主語や補語の省略などは行われにくくなる。
内容や対象を限定するために、形容詞や副詞を前に置くことが多くなる。
しかしながら、実際の文では、句の単位の語順が絶対的に固定されるわけではない。
\else
Adjustive grammar differs from immediate grammar in that there is a conscious adjustment of grammatical forms.
The most typical example is formal grammar.
It is less likely that subjects and complements are omitted.
To limit the content or object, adjectives and adverbs are often placed before nouns.
However, in actual sentences, the order of phrases is not absolutely fixed.
\fi

\ifJPN
\subsubsection{調整文法の定義}
\else
\subsubsection{Definition of Adjustive Grammar}
\fi

\ifJPN
調整文法は、時間的余裕を持って計画され、文脈や目的に基づいて慎重に調整された言語表現を生成するプロセスである。
調整文法を設計するためには、まず調整とは何かを明確に定義する必要がある。
調整は、発話の意味や形態を適応・修正する過程として捉える。
たとえば、発話の明確化や情報の修正、他者との意図の調整を行うことを意味し、「意味の明確化」「誤解の訂正」「文法的な修正」「コミュニケーションのフローの維持」といった要素が含まれる。しかし、内容の追加や説得といった言語の調整を行う際には、調整過剰点(後述)に注意する必要がある。
\else
Adjustive grammar is a process that generates language expressions that are carefully adjusted based on context and purpose, planned with temporal flexibility.
To design adjustive grammar, it is necessary to clearly define what adjustment means.
Adjustment is understood as the process of adapting and modifying the meaning or form of utterances.
For example, it refers to clarifying utterances, correcting information, and aligning intentions with others, including elements such as ``clarification of meaning,'' ``correction of misunderstandings,'' ``grammatical corrections,'' and ``maintenance of communication flow.'' However, when making adjustments such as adding content or persuasion, it is important to be cautious of the adjustment excess point (to be discussed later).
\fi

\ifJPN
\subsubsection{調整文法の基本構造}
\else
\subsubsection{Basic Structure of Adjustive Grammar}
\fi

\ifJPN
調整文法の基本的な構造は、以下のような流れに沿ったフォーマットを設計する。
まず、初期発話について、初期発話における情報の提供(述部で終わる、基本的な事実の提供など)。
調整の開始としては、初期発話が意図する意味や構造に対する認識の変化があった場合、その修正の開始を示す。
調整文法は、この「修正」のタイミングを明確に記録し、何が修正されたのかを示す。
調整の過程(再構築)については、修正が必要な場合、どのような方法で調整が行われたのか(文法的な再構築、情報の追加、発話の再順序化など)を記録する。
調整後の発話(修正発話)については、調整を経て完成された発話の結果、どのような意味の整合性が保たれたかを示す。
このような流れを具体的にフォーマット化することで、調整文法がどのように機能するかがより明確になる。
\else
The basic structure of adjustive grammar is designed along the following flow:
First, regarding the initial utterance, it provides information in the initial utterance (ending with a predicate, providing basic facts, etc.).
As the start of adjustment, if there is a change in recognition regarding the meaning or structure intended by the initial utterance, it indicates the beginning of that correction.
Adjustive grammar clearly records this timing of ``correction'' and indicates what has been corrected.
Regarding the process of adjustment (reconstruction), if correction is necessary, it records how the adjustment was made (grammatical reconstruction, addition of information, reordering of utterances, etc.).
After adjustment, the final utterance (corrected utterance) indicates what kind of semantic consistency was maintained as a result of the completed utterance after adjustment.
By concretely formatting this flow, it becomes clearer how adjustive grammar functions.
\fi

\ifJPN
\subsubsection{調整過程の理論的背景}
\else
\subsubsection{Theoretical Background of the Adjustment Process}
\fi

\ifJPN
調整文法のフォーマットを理論的に支えるために、調整がどのような認知的・社会的なプロセスであるかを示す必要がある。
たとえば、調整が「意思疎通の不確実性を減少させる」過程であることや「相手の理解を前提とした適応的な修正である」ことを言語に反映させることが重要である。
認知的要素として、調整は、発話の意味を相手に正確に伝えたり、誤解を解いたりする認知的なプロセスであるため、その過程を動的であり、調整を行うこと自体は一瞬である。
社会的要素としては、調整は会話における社会的な適応でもあり、相手との共同作業であるため、対話者間の意図の調整や確認も関係する。
これらが言語の調整に関わる限りにおいて、調整文法は、認知的・社会的なプロセスを言語表現に反映するためのフォーマットとして機能する。
しかしながら、内容の調整に関わると、その瞬間から調整過剰点に達する可能性があるため、調整文法の設計には注意が必要である。
\else
To theoretically support the format of adjustive grammar, it is necessary to demonstrate what cognitive and social processes adjustment involves.
For example, it is important to reflect in language that adjustment is a process that ``reduces uncertainty in communication'' and that it is an ``adaptive correction based on the assumption of the listener's understanding.''
As a cognitive element, adjustment is a cognitive process that accurately conveys the meaning of utterances to the listener or resolves misunderstandings, making the process dynamic, with the act of adjustment itself being instantaneous.
As a social element, adjustment is also a social adaptation in conversation and a collaborative effort with the listener, involving the alignment and confirmation of intentions between interlocutors.
As long as these are related to language adjustment, adjustive grammar functions as a format to reflect cognitive and social processes in language expression.
However, when it comes to content adjustment, there is a possibility of reaching the adjustment excess point from that moment, so care must be taken in the design of adjustive grammar.
\fi

\ifJPN
\subsubsection{調整文法のフィードバックループ}
\else
  \subsubsection{Feedback Loop in Adjustive Grammar}
\fi

\ifJPN
調整文法には、フィードバックループの概念を組み込むことも考えられる。
発話後にフィードバックを受け、その内容に基づいて修正が行われるプロセスを反映させる。
これにより、調整が単なる修正の一過程ではなく、動的に発展していく過程であることを示すことができる。
例えば、発話$\rightarrow$理解$\rightarrow$再確認$\rightarrow$調整という一連の流れにおいて、各段階における発話の変更や修正を記録する。
\else
In adjustive grammar, it is also possible to incorporate the concept of a feedback loop.
This reflects the process of receiving feedback after utterance and making corrections based on that content.
This allows us to show that adjustment is not just a one-time correction but a dynamically evolving process.
For example, in a series of flows such as utterance $\rightarrow$ understanding $\rightarrow$ re-confirmation $\rightarrow$ adjustment, we record changes or corrections to utterances at each stage.
\fi

\ifJPN
\subsubsection{調整文法の具体例}
\else
\subsubsection{Specific Examples of Adjustive Grammar}
\fi

\ifJPN
例えば、ニュースの原稿では述部で終わるという単純なルールがある一方、調整文法ではその後に追加的な情報や修正が行われるケースがある。
この場合、調整文法では以下のような構造を取ることが考える。
初期発話には「地震が発生した」という事実が述べられる。
調整の開始時では、「...震源地は日本の東部である」という情報が追加される。
調整の過程として、情報追加による意味の明確化が記載される。
調整後の発話として「地震が発生されました。震源地は日本の東部であり、マグニチュードは6.5と予測されています」という修正発話が生成される。
\else
For example, while there is a simple rule in news scripts that ends with a predicate, in adjustive grammar, there are cases where additional information or corrections are made afterward.
In this case, the following structure can be considered in adjustive grammar:
The initial utterance states the fact that ``an earthquake has occurred.''
At the start of adjustment, additional information such as ``the epicenter is in eastern Japan'' is added.
The process of adjustment records the clarification of meaning through the addition of information.
The adjusted utterance generated afterward is ``An earthquake has occurred. The epicenter is in eastern Japan, and the magnitude is predicted to be 6.5.''
\fi

\ifJPN
\subsubsection{調整文法のルール設計}
\else
  \subsubsection{Rule Design for Adjustive Grammar}
\fi

\ifJPN
修正ルールとしては、調整を行うタイミングや方法を定義(情報追加、言い換え、確認質問など)。
変更の種類として意味の修正、語順の変更、文法構造の修正などを記述する。
このように、調整文法のフォーマットは、理論的な背景に基づいて「適応的修正」の過程を明確に表現することが重要であり、各段階の具体的な記述方法、その過程における動的な変化の記述を一律に行うことが重要である。
\else
As for modification rules, it is important to define the timing and methods for making adjustments (such as adding information, rephrasing, or confirmation questions).
Types of changes include semantic corrections, changes in word order, and modifications to grammatical structures.
Thus, the format of adjustive grammar is crucial for clearly expressing the process of ``adaptive correction'' based on theoretical backgrounds, and it is important to uniformly describe specific methods for each stage and the dynamic changes in that process.
\fi

\ifJPN
\subsubsection{基本モデル}
\else
  \subsubsection{Basic Model}
\fi

\ifJPN
調整文法は以下の関数として定義される。
\else
Adjustive grammar is defined by the following function.
\fi

\[
f_{\text{AG}} : (C, M, P, T) \to O
\]

\ifJPN
ただし、\( C \) はコンテキスト(状況や文脈)を表し、話題、聞き手の背景知識、社会的関係などが含まれる。\( M \) は言語的知識(語彙、文法構造)を指し、書き言葉特有の構文や表現選択が挙げられる。\( P \) は計画(目的や調整戦略)を意味し、発話の意図(説得、説明)やスタイル(フォーマル、カジュアル)などが含まれる。\( T \) は時間的余裕を表し、これが大きいほど調整可能な幅が広がる。最後に、\( O \) は出力(生成された言語表現)を意味し、精緻化された発話や推敲された文章が含まれる。
\else
Here, \( C \) represents context (situation and background), which includes topics, the listener's background knowledge, and social relationships. \( M \) refers to linguistic knowledge (vocabulary, grammatical structures), including syntax and expression choices specific to written language. \( P \) signifies planning (purpose and adjustment strategies), encompassing the intention of utterances (persuasion, explanation) and style (formal, casual). \( T \) represents temporal flexibility, indicating that the larger it is, the wider the scope for adjustment. Finally, \( O \) signifies output (generated language expression), including refined utterances or polished texts.
\fi

\ifJPN
\subsubsection{計画性の条件}
\else
\subsubsection{Condition of Planning}
\fi

\ifJPN
調整文法では、発話計画が重要な要素です。計画の具体性は次の関数で表される。
\else
In adjustive grammar, the utterance plan is an important element. The specificity of the plan is expressed by the following function.
\fi

\[
P = g(C, M)
\]

\ifJPN
ただし、
\( g \)はコンテキスト \( C \) と言語知識 \( M \) に基づいて計画を生成する関数である。
計画 \( P \) が生成されることで、調整された言語表現が可能になる。
\else
Here, \( g \) is a function that generates plans based on context \( C \) and linguistic knowledge \( M \).
The generation of plan \( P \) enables the production of adjusted language expressions.
\fi

\ifJPN
\subsubsection{時間的余裕}
\else
\subsubsection{Temporal Flexibility}
\fi

\ifJPN
調整文法では、時間的余裕 \( T \) が制約ではなく、生成の質を高める要因として機能する。
このため、処理時間 \( t_{\text{pr}} \) は次の条件を満たす必要がある。
\else
In adjustive grammar, temporal flexibility \( T \) functions not as a constraint but as a factor that enhances the quality of generation.
Therefore, the processing time \( t_{\text{pr}} \) must satisfy the following condition.
\fi

\[
t_{\text{pr}} > t_{\text{th}}
\]

\ifJPN
ただし、
\( t_{\text{th}} \)は、即時文法に必要な処理時間の閾値で、
時間的余裕が多いほど、調整の精度や深度が増す。
\else
  Here, \( t_{\text{th}} \) is the threshold for processing time required for immediate grammar, and the more temporal flexibility there is, the greater the accuracy and depth of adjustment.
\fi

\ifJPN
\subsubsection{調整の最適化}
\else
\subsubsection{Optimization of Adjustment}
\fi

\ifJPN
調整文法では、出力 \( O \) を計画 \( P \) に基づいて最適化する。
この最適化は以下の式で表わせる。
\else
In adjustive grammar, output \( O \) is optimized based on plan \( P \).
This optimization can be expressed by the following equation.
\fi

\[
\arg\max_{O} U(O \mid C, M, P, T)
\]

\ifJPN
\( U(O \mid C, M, P, T) \): 出力 \( O \) の適切性や目的達成度を表すユーティリティ関数。
\else
\( U(O \mid C, M, P, T) \): Utility function representing the appropriateness or achievement of purpose of output \( O \).
\fi

\ifJPN
\subsubsection{連続性の表現}
\else
\subsubsection{Representation of Continuity}
\fi

\ifJPN
調整文法では、コンテキストや計画に応じて連続的な調整が行われる。
この連続性を以下の微分式で示す。
\else
In adjustive grammar, continuous adjustments are made according to context and plan.
This continuity is shown by the following differential equations.
\fi

\[
\frac{\partial O}{\partial C}, \frac{\partial O}{\partial P} \neq 0
\]

\ifJPN
ただし、
\( \frac{\partial O}{\partial C} \)は、文脈変化に対する調整の感度、
\( \frac{\partial O}{\partial P} \)は、計画変更に対する調整の感度である。
\else
Here, \( \frac{\partial O}{\partial C} \) represents the sensitivity of adjustment to changes in context, and \( \frac{\partial O}{\partial P} \) represents the sensitivity of adjustment to changes in plan.
\fi

\ifJPN
\subsubsection{調整文法の全体式}
\else
\subsubsection{Overall Expression of Adjustive Grammar}
\fi

\ifJPN
調整文法の全体式は次のように記述される。
\else
The overall expression of adjustive grammar is described as follows.
\fi

\[
O = f_{\text{AG}}(C, M, P, T) \quad \text{s.t.} \quad t_{\text{pr}} > t_{\text{th}}
\]

\ifJPN
ここで、調整文法の目的は、計画や文脈に基づいて、時間的余裕を最大限に活用した精緻化された出力 \( O \) を生成することである。
\else
Here, the purpose of adjustive grammar is to generate refined output \( O \) that maximizes temporal flexibility based on plans and context.
\fi

\ifJPN
\subsubsection{調整過剰点あるいは調整の収束}
\else
\subsubsection{Adjustment Excess Point or Convergence of Adjustment}
\fi

\ifJPN
言語の使用において、話者や書き手は適切な表現を選び、発話や文章を調整する。この調整のプロセスは、単なる発話の修正ではなく、伝達の効果を最大化するための最適化である。しかし、調整には限界があり、一定の閾値を超えると、伝達の効果が向上するのではなく、むしろ低下する。この閾値を\textbf{調整過剰点(over-adjustment point)}と呼ぶ。調整過剰点を超えた場合、過度な説明による冗長性の増大や、情報の過多による内容の変質が生じ、結果として意図した伝達が妨げられる。
\else
  In language use, speakers and writers select appropriate expressions and adjust their utterances or texts. This adjustment process is not merely a correction of utterances but an optimization to maximize the effectiveness of communication. However, there are limits to adjustment, and exceeding a certain threshold does not enhance the effectiveness of communication but rather diminishes it. This threshold is referred to as the \textbf{adjustment excess point}. When the adjustment excess point is exceeded, excessive explanations lead to increased redundancy, and an overload of information results in a distortion of content, ultimately hindering the intended communication.
\fi

\ifJPN
調整文法の最適化を考える際、この調整過剰点を組み込んだ数理モデルを検討する。
ここでは、調整の度合いを定量化し、調整過剰点を閾値とするモデルを構築する。
\else
When considering the optimization of adjustive grammar, we examine a mathematical model that incorporates the adjustment excess point.
Here, we construct a model that quantifies the degree of adjustment and uses the adjustment excess point as a threshold.
\fi

\ifJPN
調整文法では、発話または文章 $O$ を、状況 $C$、言語知識 $M$、目的 $P$ 、時間 $T$ に基づいて最適化する関数として表す。
\else
In adjustive grammar, the utterance or text \( O \) is expressed as a function optimized based on situation \( C \), linguistic knowledge \( M \), purpose \( P \), and time \( T \).
\fi

\begin{equation}
    O = f_{\text{AG}}(C, M, P, T)
\end{equation}

\ifJPN
ただし、調整の度合い $c_{\text{adj}}$ は、調整過剰点 $c_{\text{max}}$ を超えない範囲でなければならない。
\else
  However, the degree of adjustment \( c_{\text{adj}} \) must not exceed the adjustment excess point \( c_{\text{max}} \).
\fi

\begin{equation}
    c_{\text{adj}} \leq c_{\text{max}}
\end{equation}

\ifJPN
ここで、
$f_{\text{AG}}(C, M, T)$ は、状況・言語知識・目的に基づく最適化関数。
$c_{\text{adj}}$ は、調整の度合いを示す。
$c_{\text{max}}$ は、調整過剰点(飽和点)を示す閾値であり、これを超えると調整が内容そのものを変質させてしまう。
この条件のもとで、調整文法の最適な適用範囲を決定する。
\else
  Here, \( f_{\text{AG}}(C, M, T) \) is the optimization function based on situation, linguistic knowledge, and purpose.
  \( c_{\text{adj}} \) indicates the degree of adjustment.
  \( c_{\text{max}} \) is the threshold indicating the adjustment excess point (saturation point), beyond which the adjustment alters the content itself.
  Under this condition, we determine the optimal application range of adjustive grammar.
\fi

\ifJPN
\subsubsection{調整過剰点の具体例と解釈}
\else
  \subsubsection{Specific Examples and Interpretation of Adjustment Excess Point}
\fi

\ifJPN
調整過剰点を超えることで、言語表現の調整が\textbf{「言語の問題」を逸脱し、「内容の問題」に変わる}点とする。以下に具体例を示す。
\else
  By exceeding the adjustment excess point, the adjustment of language expression \textbf{deviates from "language issues" and becomes "content issues."} Below are specific examples.
\fi

\ifJPN
\subsubsection*{母親が娘に宛てた手紙}
\else
  \subsubsection*{Letter from a Mother to Her Daughter}
\fi
\ifJPN
母の想いや願い、心配が込められる。しかし、細かいことを何度も繰り返し書くと、娘にとっては負担になり、むしろ読まないほうがよいとすら思われるかもしれない。この場合、適切な調整を超えてしまい、調整過剰点に達していることになる。
\else
  The letter is filled with the mother's feelings, wishes, and concerns. However, if she repeatedly writes about trivial matters, it may become a burden for the daughter, who might even think it better not to read it at all. In this case, it has exceeded appropriate adjustment and reached the adjustment excess point.
\fi

\ifJPN
\subsubsection*{演劇のシナリオ}
\else
  \subsubsection*{Theater Script}
\fi

\ifJPN
観客の理解、役者の発話のしやすさ、内容の伝達のバランスを考慮して台詞が書かれる。しかし、観客に理解しやすくしようとしすぎると、すべてを言葉で説明してしまい、演劇の芸術性が損なわれる。ここでも、調整過剰点を超えることで、もはや言語の問題ではなく、演劇の本質に関わる問題へと変化してしまう。
\else
  The script is written considering the audience's understanding, ease of speech for the actors, and the balance of content delivery. However, if it tries too hard to make it understandable for the audience, it ends up explaining everything in words, which undermines the artistic quality of the theater. Here too, exceeding the adjustment excess point transforms the issue from a language problem to one that pertains to the essence of theater.
\fi

\ifJPN
\subsubsection*{法律文書}
\else
  \subsubsection*{Legal Documents}
\fi
\ifJPN
法律は解釈のブレが生じないように細かく調整されるが、それでも一意の解釈にはならず、最終的には法廷での審議や判決によって決定される。ここでは、言語の調整だけでは問題を解決できず、法律の適用という別の手段に頼らざるを得なくなっている。これは、調整過剰点を超えた典型的な例である。
\else
  Laws are finely adjusted to avoid interpretative discrepancies, but they still do not lead to a unique interpretation, ultimately being decided through court discussions and judgments. Here, language adjustment alone cannot resolve the issue, and it becomes necessary to rely on another means of legal application. This is a typical example of exceeding the adjustment excess point.
\fi

\ifJPN
\subsubsection{調整過剰点を考慮した最適化の必要性}
\else
  \subsubsection{Need for Optimization Considering Adjustment Excess Point}
\fi

\ifJPN
調整文法を適用する際、調整過剰点を考慮することで、言語表現の最適化を適切な範囲で行うことができる。もし調整過剰点を無視して調整を続けると、発話や文章が本来の目的を逸脱し、むしろ伝達の効果を損なってしまう。
調整過剰点を適切に設定することで、調整文法は「言語の枠内での最適化」にとどまり、内容に踏み込みすぎることを防ぐことができる。この点で、調整文法のルールは、即時文法と同様に体系化されるべきである。
\else
  By considering the adjustment excess point when applying adjustive grammar, it is possible to optimize language expression within an appropriate range. If adjustments continue without regard to the adjustment excess point, utterances or texts may deviate from their original purpose and actually undermine the effectiveness of communication.
By appropriately setting the adjustment excess point, adjustive grammar can remain within the bounds of "optimization within language" and prevent overstepping into content issues. In this regard, the rules of adjustive grammar should be systematized similarly to immediate grammar.
\fi

\ifJPN
調整文法は、言語表現の最適化を目的とするが、調整過剰点を超えると、言語の問題を逸脱し、内容の問題に変わる。
したがって、調整文法を適用する際には、調整過剰点を閾値とし、その手前で調整を止める必要がある。このモデルを導入することで、調整文法の適用範囲を明確にし、言語表現の最適化を適切な範囲で行うことができる。
\else
  Adjustive grammar aims to optimize language expression, but exceeding the adjustment excess point deviates from language issues and becomes a content issue.
  Therefore, when applying adjustive grammar, it is necessary to set the adjustment excess point as a threshold and stop adjustments before reaching it. By introducing this model, we can clarify the application range of adjustive grammar and optimize language expression within an appropriate scope.
\fi

\ifJPN
\subsubsection{定義の厳格化}
\else
\subsubsection{Strict Definition}
\fi

\ifJPN
即時文法と調整文法それぞれの基本的な定義と特徴を強調し、異なる観点からアプローチすることを明確にすることが重要である。
たとえば、即時文法は「その場で反応的に発話が生成されるプロセス」、調整文法は「言語的な適応や修正を行うプロセス」と捉えれば、フォーカスする側面が異なることがわかりやすい。
即時文法の特徴は「即時的で反射的な選択」であり、調整文法の特徴は「意図的な修正や確認」であり、両者は論理的に区別される側面がある。
\else
It is important to emphasize the basic definitions and characteristics of immediate grammar and adjustive grammar, and to approach them from different perspectives.
For example, if immediate grammar is understood as ``a process in which speech is generated reactively on the spot'' and adjustive grammar is understood as ``a process of linguistic adaptation and correction,'' it is easy to understand that the focus is different.
The characteristic of immediate grammar is ``immediate and reflexive selection,'' and the characteristic of adjustive grammar is ``intentional correction and confirmation,'' and there are logically distinguishable aspects between the two.
\fi

\ifJPN
  \section{概念図}% 20250303
\else
  \section{Concept Diagram}% 20250303
\fi

\ifJPN
ここではプロセス文法モデル全体を見渡すと同時に即時文法と調整文法の関係を示す概念図を示す(図\ref{fig:boundary-j})。
\else 
  Here, we show a concept diagram that shows the relationship between immediate grammar and adjustive grammar while looking at the entire Process Grammar Model (Figure \ref{fig:boundary}).
\fi


\begin{figure}[htb]\centering\small
\ifTIKZFIG
\begin{tikzpicture}[node distance=35mm]\small
    % 時間軸
  \ifJPN
    \draw[thick,->,>=stealth,line width=.3mm] (0,0) -- (14,0) node[anchor=north] {時間(Time)};
  \else
    \draw[thick,->,>=stealth,line width=.3mm] (0,0) -- (14,0) node[anchor=north] {Time};
  \fi

    % 上:即時文法の適用範囲(破線)
    \draw[thick,draw=red] (2.05,.08) -- (6.95,.08);
    % 下:調整文法の適用範囲(細い実線)
    \draw[thick,draw=blue] (7.05,-.08) -- (11.95,-.08);

    % 限界のノード(斜め配置・改行対応)
  \ifJPN
    \node[rotate=0, anchor=south] at (2,1) 
      {\parbox{4cm}{\centering 前限界 \\ Pre-boundary}};
  \else
    \node[rotate=0, anchor=south] at (2,1.5) 
      {\parbox{4cm}{\centering Pre-boundary}};
  \fi

  \ifJPN
    \node[rotate=0, anchor=south] at (7,1) 
      {\parbox{4cm}{\centering 後限界 \\ =調整文法の前限界 \\ Post-boundary}};
  \else
    \node[rotate=0, anchor=south] at (7,1.5) 
      {\parbox{4cm}{\centering Post-boundary \\ Pre-boundary of Adjustive Grammar}};
  \fi

  \ifJPN
    \node[rotate=0, anchor=south] at (12,1) 
      {\parbox{4cm}{\centering 調整文法の後限界 \\ =調整飽和点 \\ Saturation Point}};
  \else
    \node[rotate=0, anchor=south] at (12,1.5) 
      {\parbox{5.5cm}{\centering Post-boundary of Adjustive Grammar \\ =Adjustive Saturation Point}};
  \fi

    % 範囲のラベル
  \ifJPN
    \node at (4.5,.6){\parbox{34mm}{\centering 即時文法の適用範囲 \\ Immediate Grammar}};
  \else
    \node at (4.5,.6){\parbox{34mm}{\centering Application Range of Immediate Grammar}};
  \fi

  \ifJPN
    \node at (9.5,.6){\parbox{34mm}{\centering 調整文法の適用範囲 \\ Adjustive Grammar}};
  \else
    \node at (9.5,.6){\parbox{34mm}{\centering Application Range of Adjustive Grammar}};
  \fi

    % フェーズの区切り(縦破線)
  \ifJPN
    \draw[dashed,<-,>=stealth,line width=.3mm] ( 2,.16) -- ( 2,1.0);
    \draw[dashed,<-,>=stealth,line width=.3mm] ( 7,.16) -- ( 7,1.0);
    \draw[dashed,<-,>=stealth,line width=.3mm] (12,.03) -- (12,1.0);
  \else
    \draw[dashed,<-,>=stealth,line width=.3mm] ( 2,.16) -- ( 2,1.5);
    \draw[dashed,<-,>=stealth,line width=.3mm] ( 7,.16) -- ( 7,1.5);
    \draw[dashed,<-,>=stealth,line width=.3mm] (12,.03) -- (12,1.5);
  \fi


% 時間軸上にマーカーを追加
\fill (2,.08) circle (2pt);  % 前限界(黒丸)
\draw[fill=white] (7,.08) circle (2pt);  % 後限界(白丸 = 即時文法にとって)
\fill (7,-.08) circle (2pt);  % 後限界(黒丸 = 調整文法にとって)
\fill (12,-.08) circle (2pt); % 調整飽和点(黒丸)

\end{tikzpicture}
\else
    \includegraphics[width=0.95\textwidth]{pgm-fig-concept.pdf}
\fi
\ifJPN
\caption{即時文法と調整文法の限界}\label{fig:boundary-j}
\else
\caption{Boundary between Immediate Grammar and Adjustive Grammar}\label{fig:boundary}
\fi
\end{figure}


%\begin{figure}[htb]\centering\small
%    \includegraphics[width=0.95\textwidth]{boundary.pdf}
%    \caption{即時文法と調整文法の限界}\label{fig:boundary}
%\end{figure}

\ifJPN
\subsection{即時文法の前限界と後限界の定義}
\else
\subsection{Definition of Pre-boundary and Post-boundary of Immediate Grammar}
\fi

\ifJPN
前限界(Pre-boundary)は、即時文法の適用が始まる点であり、発話が即時的な処理として立ち上がる時点を示す。  
即時文法の適用範囲は、前限界から後限界までの区間である。  
\else
The Pre-boundary is the point at which the application of immediate grammar begins, indicating the moment when speech starts to operate as an immediate process.
The application range of immediate grammar is the interval from the Pre-boundary to the Post-boundary.
\fi

\ifJPN
後限界(Post-boundary)は、即時文法の適用範囲の終わりを示す点であり、即時的な生成・連鎖が維持されにくくなる限界である。  
ただし、即時文法の後限界は必ずしも調整文法の前限界と一致するとは限らない。  
実際には、即時文法と調整文法の適用範囲が時間的に重複することがあり、両者が併走する移行域が生じる場合もある。  
\else
The Post-boundary indicates the end of the application range of immediate grammar, representing the point beyond which immediate generation and chaining become difficult to sustain.
However, the Post-boundary of immediate grammar does not necessarily coincide with the Pre-boundary of adjustive grammar.
In practice, the application ranges of immediate grammar and adjustive grammar may overlap in time, yielding a transitional zone in which both operate concurrently.
\fi

\ifJPN
\subsection{調整文法の適用範囲と調整飽和点}
\else
\subsection{Application Range of Adjustive Grammar and Adjustive Saturation Point}
\fi

\ifJPN
調整文法の前限界(Pre-boundary)は、調整文法の適用が始まる点を示す。  
調整文法は、発話や文章に対して、言い直し・付け足し・語の選び直し・構造の整理などの調整を行う際に適用される。  
このとき、調整文法の開始点は必ずしも即時文法の後限界と一致するとは限らず、両者が時間的に重複する場合もある。  
\else
The Pre-boundary of adjustive grammar indicates the point at which the application of adjustive grammar begins.
Adjustive grammar is applied when an utterance or a text is adjusted through operations such as rephrasing, adding clarifying material, reconsidering lexical choices, or reorganizing structure.
In this sense, the starting point of adjustive grammar does not necessarily coincide with the Post-boundary of immediate grammar, and the two may overlap in time.
\fi

\ifJPN
調整文法の後限界(Post-boundary)は、ある一定の調整を行った後、それ以上の改善が見込めなくなる点を指す。  
この点を調整飽和点(Saturation Point)と呼び、調整文法の適用範囲の終わりを示す。  
\else
The Post-boundary of adjustive grammar indicates the point at which further improvement is no longer expected after a certain amount of adjustment.
This point is called the adjustive saturation point, indicating the end of the application range of adjustive grammar.
\fi

\ifJPN
\subsection{即時文法と調整文法の境界}
\else
\subsection{Boundary between Immediate Grammar and Adjustive Grammar}
\fi

\ifJPN
即時文法と調整文法のあいだには境界が存在するが、その境界は固定的ではない。  
発話は、調整された表現を反復してリハーサルする、記憶として定着させる、定型表現として共有する、といった過程を経ることで「熟練」される。  
このようにして熟練された表現は、即時的に発話されることもある。  
これは、即時文法と調整文法が互いに独立した領域として完全に分離しているのではなく、経験や熟練によって相互に影響し合うことを示している。  
したがって、発話は即時文法と調整文法のいずれかに一度決まって固定されるのではなく、状況や目的に応じて両者のあいだを移行しうる。  
\else
There is a boundary between immediate grammar and adjustive grammar, but this boundary is not fixed.
An utterance may become "skilled" through processes such as repeated rehearsal of adjusted expressions, stabilization in memory, or conventionalization as shared formulaic patterns.
Such skilled expressions may be produced immediately in real-time interaction.
This suggests that immediate grammar and adjustive grammar are not completely separated as independent domains, but mutually influence each other through experience and skill.
Therefore, speech is not permanently assigned to either immediate grammar or adjustive grammar, but may shift between them depending on context and communicative goals.
\fi

\ifJPN
プロセス文法モデルでは、「即時文法」は直感的・瞬発的な発話の側面を扱い、「調整文法」は意識的・分析的な調整の側面を扱う。  
ただし両者は排他的に切り替わるのではなく、同じ発話過程の中で時間的に重複し、併走する場合もある。  
また、熟練によって調整文法による操作が自動化され、即時的な生成の中に取り込まれることがある。  
このような変化は、プロセス文法モデルが固定的な文法体系ではなく、時間と経験の中で変化する「プロセス(過程)」を記述する動的モデルであることを示している。  
\else
In the Process Grammar Model, immediate grammar captures the intuitive and instantaneous aspects of speech, whereas adjustive grammar captures the conscious and analytical aspects of adjustment.
However, the two do not switch in a strictly exclusive manner; they may overlap in time and operate concurrently within the same utterance process.
Furthermore, through skill acquisition, operations that were once handled by adjustive grammar may become automatized and incorporated into immediate generation.
Such changes indicate that the Process Grammar Model is not a fixed grammatical system, but a dynamic model describing a time-dependent "process."
\fi

\ifJPN
  \subsection{即時文法と調整文法の連続体}
\else
  \subsection{Continuum of Immediate Grammar and Adjustive Grammar}
\fi

\ifJPN
即時文法と調整文法の関係は、単純な二分法ではなく、状況・目的・経験に応じて両者の関与の仕方が変化するという意味で「連続体」として捉えられる。  
ここでいう連続体とは、両者のあいだに第三の文法が存在するという意味ではなく、発話過程の中で即時文法と調整文法が時間的に重複し、また相互の負担配分が変動しうるという性質を指す。  
たとえば語学学習では、初学者が文を組み立てる際に調整文法(システム2)の関与が大きいが、訓練によって頻繁に用いられるパターンが自動化されると、即時的な出力(システム1の使用)が可能となる。  
このような熟練の過程により、同一の課題であってもシステム1とシステム2の役割分担は変化しうる。  
したがって、ある時点では調整文法が担っていた操作が、別の時点では即時的な生成の中に取り込まれて機能する場合がある。  
この可変性が、プロセス文法モデルにおける「連続体」の要点であり、即時文法と調整文法の相補的かつ不可分な関係を示す。  
\else
The relationship between immediate grammar and adjustive grammar is not a simple binary opposition, but can be regarded as a "continuum" in the sense that their involvement varies depending on context, goals, and experience.
In this study, the continuum does not mean that a third grammar exists between them. Rather, it refers to the fact that immediate grammar and adjustive grammar may overlap in time within the utterance process, and that their functional allocation may shift dynamically.
For example, in language learning, beginners rely heavily on adjustive grammar (System 2) when constructing sentences, but through training, frequently used patterns may become automatized, enabling immediate output (the use of System 1).
Through such skill acquisition, the division of labor between System 1 and System 2 can change even for the same task.
Accordingly, operations once handled by adjustive grammar may later be incorporated into immediate generation.
This variability constitutes the core of the "continuum" in the Process Grammar Model, demonstrating the complementary and inseparable relationship between immediate grammar and adjustive grammar.
\fi

\ifJPN
ただし、即時文法と調整文法は同一形式の強弱ではなく、それぞれ異なるフォーマットとして両極をなす。  
実際の言語使用では両者が相互作用することが多く、同一の発話過程の中で即時的な選択と、その直後の調整が連動して現れる場合がある。  
即時文法が言語使用の瞬間的側面に焦点を当てるのに対し、調整文法は適応的側面と調整過程に焦点を当てる。  
両者を統一的に扱う記述フォーマットを用意することは、言語使用のダイナミズムを示す上で有益である。  
しかしその一方で、記述を統一しすぎると即時文法と調整文法の概念規定を曖昧にするおそれがあるため、両者が同一化しないように区別を保持する論理が必要である。  
\else
However, immediate grammar and adjustive grammar are not simply weaker and stronger versions of the same format; they constitute different formats positioned at opposite poles.
In actual language use, they often interact, and immediate selection in real-time conversation may be closely coupled with subsequent adjustment within the same utterance process.
While immediate grammar focuses on the instantaneous aspects of language use, adjustive grammar focuses on adaptive aspects and the adjustment process.
Providing a unified descriptive format for the two may be useful for demonstrating the dynamism of language use.
At the same time, excessive unification may blur the conceptual definitions of immediate grammar and adjustive grammar; therefore, a logic that preserves their distinction is required.
\fi

\ifJPN
一般に速い処理はシステム1の使用として扱われやすいが、それは必ずしも生得的にシステム1由来であることを意味しない。  
経験や反復によって、当初はシステム2によって遂行されていた操作が自動化され、速い処理として実行されるようになる場合がある\autocite{Evans2008}。  
本来熟考が必要な課題であっても、訓練や反復によって処理が自動化され、結果としてシステム1的な処理として現れることがある。  
語学学習においても、当初は即座にできなかった文構築が、熟達により即座に使用可能になる例が見られる。  
一方、新しい語や複雑な手順を説明する場面では、誤解を排除するためにシステム2による慎重な調整が必要となるだろう。  
\else
Fast processing is often treated as the use of System 1, but this does not necessarily mean that it is innately derived from System 1.
Through experience and repetition, operations that were initially carried out by System 2 may become automatized and executed as fast processing.\autocite{Evans2008}
Even for problems that would normally require careful deliberation, training and repeated practice can lead to automatized processing that appears System-1-like.
In language learning, sentence construction that was not immediately available at an early stage may become immediately usable after mastery.
On the other hand, when explaining new words or complex procedures, careful System 2 adjustment is likely to be required in order to avoid misunderstanding.
\fi

\ifJPN
  \section{襷掛け効果}
\else
  \section{Tasuki-gake/cross-directional Effect}
\fi



\newcommand{\TasukiGakeEffectJa}{
\textbf{襷掛け効果(tasuki-gake effect)}とは、「即時文法」と「調整文法」が本来の場面を越えて交差的に使用されることによって生じる文体的効果を指す。
即時文法が調整文法的媒体(例:小説、スピーチ)に用いられると、発話に速度や臨場感が加わる。
逆に、調整文法が即時の発話場面で用いられると、丁寧さや形式性が帯びる。
このような交差的な使用によって、言語表現にリズムや文体的奥行きが生まれる現象を「襷掛け効果」と呼ぶ。
}

\newcommand{\TasukiGakeEffectEn}{
\textbf{Tasuki-gake effect} refers to a stylistic phenomenon where Immediate Grammar and Adjustive Grammar are used in contexts that cross their typical domains of application.
When Immediate Grammar appears in a normally Adjustive context (such as a novel or formal speech), it injects liveliness and immediacy.
Conversely, when Adjustive Grammar is used in spontaneous utterances, it adds politeness or formality.
This interplay between grammatical modes creates a distinctive stylistic depth, referred to as the tasuki-gake effect.
}


\ifJPN
  \TasukiGakeEffectJa
\else
  \TasukiGakeEffectEn
\fi


\ifJPN
  \begin{figure}[htb]\centering\small
    \includegraphics[width=0.25\textwidth]{./figures/tasuki-gake01.png}
    \caption{襷掛け: 和服の袂が作業をする際に邪魔になるので、長い布紐で、背中がクロスになるように袂を手繰り寄せる肩紐の掛け方。}\label{fig:tasuki-gake-picture-j}
  \end{figure}
\else
  \begin{figure}[htb]\centering\small
    \includegraphics[width=0.25\textwidth]{./figures/tasuki-gake01.png}
    \caption{Tasuki-gake: a method of tying a long cloth strap around the back to pull the sleeves of a kimono together, so that they cross over the back. This is done to prevent the sleeves from getting in the way while working.}\label{fig:tasuki-gake-picture}
  \end{figure}
\fi

\ifJPN
  \begin{figure}[htb]\centering\small
    \includegraphics[width=0.95\textwidth]{tasuki-gake.pdf}
    \caption{襷掛け効果: プロセス文法における効果の交差的構造。
    日本の伝統的な「たすき掛け」は、前と後ろで左右の紐が交差する形式をとる。これは、即時的な発話が調整的効果を、調整された発話が自然で親しみやすい印象を生むという、効果と使用の関係の「ねじれ」を視覚的に表す比喩である。
    }\label{fig:tasuki-gake-j}
  \end{figure}
\else
  \begin{figure}[htb]\centering\small
    \includegraphics[width=0.95\textwidth]{tasuki-gake.pdf}
    \caption{Tasuki-gake Effect: A Visual Metaphor for Crossed Effects in Process Grammar. The diagram illustrates the traditional way of tying a tasuki sash. Just as the front-left is tied to the back-right, and vice versa, the relationship between utterance types and their perceived effects often crosses over. Immediate utterances can produce polite effects, while adjusted utterances may evoke natural or spontaneous impressions.
    }\label{fig:tasuki-gake}
  \end{figure}
\fi



\ifJPN
本モデルでは、言語の使用はまず時間的基盤において二種の発話(utterance)に分けられる。すなわち、即時に発せられる「Immediate Utterance(即時発話)」と、熟考や調整の後に発せられる「Adjustive Utterance(調整発話)」である。
この二種の発話は、それぞれ異なる文法的拘束を受ける。Immediate Utterance は直感的な発話の連鎖を導く「Immediate Grammar(即時文法)」に支えられ、Adjustive Utterance は判断や推敲を経た表現の形成を担う「Adjustive Grammar(調整文法)」に支えられる。

ここから生まれる言語表現は、それぞれ「Immediate Expression(即時表現)」「Adjustive Expression(調整表現)」と呼ばれる。
ただし本モデルにおいて重要なのは、即時表現と調整表現が、それぞれ一定の効果に固定されるわけではなく、使用される状況によって効果が交差しうるという点である。

Adjustive Expression が訓練や定型化を経て即時に発話されたとき、それは公的かつ丁寧な印象を与える「Polite and Public Effect(丁寧・公的効果)」をもたらしうる。
一方、Immediate Expression が意図的に小説やインタビュー記事などに組み込まれた場合には、生き生きとした会話的な印象を与える「Natural and Spoken Effect(自然・話しことば的効果)」を発揮しうる。

このように、発話の時間的性質(即時/調整)と、表現の出自(即時表現/調整表現)が生む効果のあいだには交差関係(襷掛け)が存在しており、単純な二分法では捉えきれない表現の機能と運用の豊かさを可視化する構造となっている(図 \ref{fig:tasuki-gake-picture-j}, \ref{fig:tasuki-gake-j})。

\else

In this model, language use is fundamentally categorized by two temporal types of utterance:
Immediate Utterance, which is produced spontaneously in real time, and
Adjustive Utterance, which is produced after deliberation and adjustment.
Each type of utterance is governed by a distinct grammatical system.
Immediate Utterance operates under Immediate Grammar, which facilitates intuitive real-time chaining of expressions.
Adjustive Utterance, in contrast, is governed by Adjustive Grammar, which supports expressions formed through conscious revision and restructuring.

The expressions derived from these grammars are referred to as
Immediate Expression and Adjustive Expression, respectively.
A crucial feature of this model, however, is that these expressions are not tied to fixed effects.
Rather, depending on how they are deployed, they may yield crossed (tasuki-gake) effects.

When an Adjustive Expression is highly practiced, formulaic, or routinized for immediate delivery,
it may give rise to the Polite and Public Effect, creating a sense of formality, refinement, or social appropriateness.
Conversely, when an Immediate Expression is intentionally employed in written or scripted media, such as novels or interview articles,
it may produce the Natural and Spoken Effect, evoking a vivid, conversational, and lifelike tone.

Thus, there is a crossed (tasuki-gake) relationship between the temporal nature of utterance and the effect of expression.
This structure makes visible the richness of expressive function and practical use that goes beyond a simple binary model (Figure \ref{fig:tasuki-gake-picture} and \ref{fig:tasuki-gake}).
\fi

\ifJPN
\section{記述の方法}
\else
\section{Description Method}
\fi

\ifJPN
\subsection{適切なフォーマットの設計}
\else
\subsection{Design of Appropriate Format}
\fi

\ifJPN
即時文法と調整文法を関係づける際には、それぞれが担う役割を反映したフォーマットをデザインする。
このフォーマットでは、各プロセスがどのタイミングでどのように発動するかを示す「プロセスのフロー」を明確に示す。
たとえば、即時文法では「反応的発話」の発生時点を明記し、その後に調整文法が適用される「修正」や「確認」のタイミングを示すことによって、異なるフォーマットを維持しつつ、両者の補完を視覚的に表現できる。
\else
When linking immediate grammar and adjustive grammar, design a format that reflects the roles they play.
In this format, clearly show the ``flow of processes'' that indicates when and how each process is triggered.
For example, in immediate grammar, clearly indicate the timing of the occurrence of ``reactive speech,'' and by indicating the timing of ``correction'' and ``confirmation'' to be applied later, it is possible to visually express the complementarity of the two while maintaining different formats.
\fi

\ifJPN
\subsection{区別のメタレベルでの検討}
\else
\subsection{Consideration at the Meta-level of Distinction}
\fi
  
\ifJPN
即時文法と調整文法は一見するとつながっているように見えるかもしれないが、実はそれぞれが異なる認知的・社会的な機能を持つ。
この点において両者が異なる状況や認知負荷の下で機能するかどうかを観察することが課題である。
たとえば、即時文法が「認知的な即時反応」を、調整文法が「複雑な言語調整を伴う認知過程」を反映しているならば、それぞれの理論的枠組みを区別するロジックが自ずと確立できよう。
\else
Although immediate grammar and adjustive grammar may seem to be connected at first glance, they actually have different cognitive and social functions.
It is a challenge to observe whether they function under different situations and cognitive loads.
For example, if immediate grammar reflects ``cognitive immediate responses'' and adjustive grammar reflects ``cognitive processes involving complex language adjustments,'' the logic to distinguish between the two theoretical frameworks will naturally be established.
\fi

\ifJPN
\subsection{実証的データによる区別の確認}
\else
\subsection{Confirmation of Distinction by Empirical Data}
\fi

\ifJPN
理論を実証的に支えるデータを集める目的は、即時文法と調整文法が異なる文脈や使用場面での具体的な発話例からシステムを構築することである。
即時文法と調整文法の両者の特徴を実例で、実際に次元分けしたマップ上にピン止めするなどして固定し、概念化の助けにすることで両極が実在性することを示し、その上にプロセス文法が横たわっていることを示す。
実際の会話や言語データを分析し、即時的な反応と時間を要する調整の区別、またはそれらのタイミングにより、両極が曖昧になるのかどうかを確認することで、具体的に両極を整理する。
\else
The purpose of collecting data to empirically support the theory is to build a system from specific speech examples in different contexts and usage situations of immediate grammar and adjustive grammar.
By pinning down the characteristics of both immediate grammar and adjustive grammar on a map divided into dimensions with actual examples, it is possible to show that both poles exist and that the process grammar lies on top of them.
By analyzing actual conversations and language data, it is possible to confirm whether the distinction between immediate responses and time-consuming adjustments, or the timing of these, makes the two poles ambiguous, and to organize them concretely.
\fi

\ifJPN
関係づける方法としては、即時文法のフォーマットにおける「反応的な要素」を調整文法のフォーマットにおける「調整過程」とのリンクの方法を提案することが一つのアプローチになる。
両極を取り持つ実際的なデータをモデルとして位置付けることで、言語の使用における「即時性」と「調整性」が相互作用するモデルのパラメタを指定できれば、相対的な発話・表現の連続体モデルが作成できると考える。
\else
One approach is to propose a method of linking the ``reactive elements'' in the format of immediate grammar with the ``adjustment process'' in the format of adjustive grammar.
By positioning actual data that mediates both poles as a model, it is possible to specify the parameters of a model in which ``immediacy'' and ``adjustability'' in language use interact, and to create a model of a continuum of relative speech and expressions.
\fi

%- 両極の特徴を説明。
%  - 即時文法の特徴: 緊急性、直感的反応(例: 「危ない!」)。
%  - 調整文法の特徴: 慎重な推敲、フォーマルな場面での使用(例: スピーチ)。
%- 中間的な状況やグラデーションを含めたモデルを提案。
%  - 日常会話の中での即時的発話と調整的発話の組み合わせ。
%  - 例えば、初対面の相手への対応では即時性と調整性が同時に求められる。
%- 視覚化(図や表)を活用。
%  - モデルのグラフやフローチャート。
%  - 即時性から調整性への連続体を示す視覚的表現。


% 補足の提案
% 
% 1. 定義の明確化  
%    「調整」が形式に限定されることを強調する際、具体的な例を挙げると読者の理解が深まります(例: 丁寧表現の選択、文法的再構成)。
% 
% 2. 視覚化の内容を事前に予告
%    視覚化が含まれることは非常に良いですが、図や表の具体的な形式(連続体のスペクトラムや事例比較表など)を明記すると、期待感が高まります。
% 
% 3. 各セクションの間のつながり
%    「定義」と「モデル」のセクションが緊密に関連していることを示すために、「モデル提案」が定義を基に構築されている旨を冒頭で簡単に述べるのが効果的です。


% \begin{figure}[ht]\centering
%     \includegraphics[width=0.7\textwidth]{processimage.pdf}
%     \caption{
%       Process grammar model of actual language use: 
%       continuum of immediate and adjusted grammars
%     }
%     \label{fig:processgrammar}
% \end{figure}

\begin{figure*}[ht]\centering\small
\begin{tikzpicture}[scale=.8, every node/.style={align=center, font=\small}]
\draw[thick] (0,0) rectangle (10,4);
\draw[thick, dashed] (0,0) -- (10,4);
\fill[blue!20, opacity=0.6] (0,0) -- (10,4) -- (0,4) -- cycle; % Upper triangle for 即時文法
\fill[red!20, opacity=0.6] (0,0) -- (10,4) -- (10,0) -- cycle; % Lower triangle for 調整文法
\ifJPN
  \node[above left] at (4,2.5) {即時文法};
  \node[below right] at (6,1.5) {調整文法};
  \node[above right] at (4,1.6) {連続体};
  \node[above left] at (0,1.5) {高い\\即時性};
  \node[below right] at (10,2.5) {高い\\調整性};
\else
  \node[above left] at (4.2,2.5) {Immediate Grammar};
  \node[below right] at (6,1.5) {Adjustive Grammar};
  \node[above right] at (3.5,1.6) {Continuum};
  \node[above left] at (0,1.5) {High\\Immediacy};
  \node[below right] at (10,2.5) {High\\Adjustability};
\fi
\end{tikzpicture}

\ifJPN
\caption{実際の言語使用のプロセス文法モデル:即時性と調整性の連続体}
\else
\caption{Process grammar model of actual language use}
\fi

\label{fig:processgrammar}
\end{figure*}

\ifJPN
\begin{table*}[ht]\centering\small
\caption{即時文法と調整文法の特徴}
\else
\begin{table*}[ht]\centering\small
\caption{Characteristics of immediate and adjustment grammars}
\fi
\label{tab:characteristics}
\ifJPN
\begin{tabular}{lp{4.5cm}p{8.5cm}}\noalign{\hrule height .8pt}
\textbf{項目} 
  & \textbf{即時文法}
  & \textbf{調整文法} \\ \hline

\textbf{特徴} 
  & その場で瞬間的に適用される文法。
  & 言語形式が適切かを検討し、必要に応じて修正・調整を行う文法。 \\ 

\textbf{時間幅} 
  & ミリ秒から数秒で処理される。 
  & 数秒から数年まで、長い時間をかけて調整される場合もある。 \\ 

\textbf{具体例} 
  & 反射的な応答、自然発生的な会話。 
  & 記者会見の応答(調整を伴う場合)、スピーチ、法律文書の推敲。 \\ 

\textbf{調整の要素} 
  & 微小の調整、無意識的な調整のみ。 
  & 言語形式に関する意識的な調整が加えられる。 \\ 

\textbf{目的} 
  & 即座に情報を伝える。 
  & 誤解を防ぎ、正確性や適切さを確保する。 \\
\end{tabular}
\end{table*}

\else

  \begin{tabular}{p{32mm}p{53mm}p{60mm}}\noalign{\hrule height .8pt}
%  & \textbf{Immediate Grammar} & \textbf{Adjustment Grammar} \\

  \textbf{Item} 
  & \textbf{Immediate Grammar} 
  & \textbf{Adjustive Grammar} \\ \hline

  \textbf{Characteristics} 
  & Grammar applied instantaneously on the spot. Little consideration or modification of form. 
  & Grammar that considers the appropriateness of linguistic forms and makes adjustments as needed. \\

  \textbf{Time Span} 
  & Milliseconds to seconds. Processed in real-time. 
  & Seconds to years. Adjustments may take a long time. \\

  \textbf{Examples} 
  & Reflexive responses, spontaneous conversations. 
  & Press conference responses (with adjustments), speeches, editing of legal documents. \\

  \textbf{Adjustive Elements} 
  & Minimal or unconscious adjustments only. 
  & Conscious adjustments to linguistic forms. \\

  \textbf{Purpose} 
  & Immediate information delivery. 
  & Preventing misunderstandings and ensuring accuracy and appropriateness. \\
\end{tabular}
\end{table*}
\fi





\ifJPN
  \section{今後の展開}
\else
  \section{Future Directions}
\fi

\ifJPN
  \subsection{研究課題}
\else
  \subsection{Research Questions}
\fi

\begin{itemize}
  \item 
\ifJPN
即時文法と調整文法の境界とその定式化(前限界・後限界の数式化)
\else
The boundary between Immediate Grammar and Adjustive Grammar and its formalization (formulation of pre-boundary and post-boundary)
\fi

  \item 
\ifJPN
実証データの収集と分析(会話データ、書きことばデータの比較)
\else
Collection and analysis of empirical data (comparison of conversation data and written data)
\fi
\end{itemize}

\ifJPN
  \subsection{アップデートの方針}
\else
  \subsection{Update Policy}
\fi

\ifJPN
ルールベースの整理と追加
\else
Organizing and adding rules
\fi
\ifJPN
言語教育やAI(自然言語処理)への応用
\else
Application to language education and AI (natural language processing)
\fi
\ifJPN
和歌や歴史的な文献を対象にした分析を検討していく。
\else
Consideration of analysis targeting waka and historical literature.
\fi

\ifJPN
  \section{おわりに}
\else
  \section{Conclusion}
\fi

\ifJPN
本ダイジェストでは、プロセス文法モデルの概要、即時文法と調整文法の対比、理論的背景を示した。今後のアップデートでは、ルールベースの整理やデータ分析を拡充し、さらなる発展を目指す。
\else
In this digest, we have outlined the Process Grammar Model, compared Immediate Grammar and Adjustive Grammar, and discussed the theoretical background. In future updates, we will expand the rule base and data analysis to further develop the model.
\fi

\appendix
\ifJPN
\section{Q\&A セクションについて}
\else
\section{About the Q\&A Section}
\fi
\label{sec:QandA}

\ifJPN
本書では、即時文法の理論モデルを提示し、それに基づく実例や応用を通じて新たな言語理解の枠組みを探ってきた。しかし、こうした新しい理論を提示するにあたっては、読者の視点からさまざまな疑問や確認したい点が生まれることも予想される。
\else
In this book, we have presented the theoretical model of immediate grammar and explored a new framework for language understanding through examples and applications based on it. However, when introducing such a new theory, it is expected that various questions and points of confirmation will arise from the reader's perspective.
\fi

\ifJPN
この「Q\&A セクション」では、そうした疑問に対する著者の立場や補足的な考察を記録しておくことを目的とする。また、このセクションには、翻訳作業(たとえば『土佐日記』『伊勢物語』)や、日々の即時文法表現の収集作業(aeadプロジェクト)において、思考の過程で浮かび上がった疑問や発見も含めていく。
\else
This ``Q\&A Section'' aims to record the author's position and supplementary considerations regarding such questions. Additionally, this section will include questions and discoveries that have emerged during the translation work (for example, ``Tosa Nikki'' and ``Ise Monogatari'') and the daily collection of immediate grammar expressions (aead project).
\fi

\ifJPN
本文に組み込むには構成上の調整が必要となるが、それとは別に「書くたびに微妙なズレが生じる」ことを避け、思考のまとまりを維持するために、このセクションを用いる。必要に応じて本文との連携を後から再検討することを想定しており、本セクションは柔軟な蓄積と検討の場として位置づけられている。
\else
To avoid ``subtle discrepancies that arise every time I write'' and to maintain the coherence of thought, this section will be used. It is assumed that the integration with the main text will be reconsidered later as needed, and this section is positioned as a flexible space for accumulation and examination.
\fi

\ifJPN
\subsection{Q\&A: 即時文法とその理論的背景}
\else
\subsection{Q\&A: Immediate Grammar and Its Theoretical Background}
\fi

% 連番Q&A:最初のグループ
\begin{enumerate}[label=\textbf{Q\arabic*.}, leftmargin=2em]

  \item \label{qa:20250405a}
\ifJPN
  \textbf{即時文法とはどのような理論ですか? また、既存の構文理論とどのように位置づけられるのですか?}
\else
  \textbf{What kind of theory is Immediate Grammar? How is it positioned in relation to existing syntactic theories?}
\fi

\ifJPN
  \textbf{A.} 即時文法は、言語使用に見られる即時的かつ柔軟な表現形式に注目し、それらを理論的に説明しうる構造として提案されるものです。このモデルは、句構造文法のような構文中心の理論と対置されるものではなく、同等の立場において、人間の言語行動を異なる観点から記述しようとする仮説的枠組みです。
\else
  \textbf{A.} Immediate Grammar focuses on the immediate and flexible forms of expression observed in language use and proposes them as structures that can be theoretically explained. This model is not positioned in opposition to syntactic-centered theories like phrase structure grammar but is a hypothetical framework that aims to describe human language behavior from a different perspective.
\fi

  \item \label{qa:20250405b}
\ifJPN
  \textbf{即時文法は、どのような点で言語記述に貢献するのですか?}
\else
  \textbf{In what ways does Immediate Grammar contribute to language description?}
\fi

\ifJPN
  \textbf{A.} 即時文法という視点を導入することで、これまで「例外的」または「構造に乏しい」とされてきた発話が、体系的な言語使用の一形態として位置づけられるようになります。これは、即時文法がそうした発話に対して理論的な「説明のスロット」を提供するモデルであることの証とも言えます。
\else
  \textbf{A.} By introducing the perspective of Immediate Grammar, utterances that have previously been considered "exceptional" or "lacking structure" can be positioned as a systematic form of language use. This can be seen as evidence that Immediate Grammar provides a theoretical "slot for explanation" for such utterances.
\fi

\end{enumerate}

\ifJPN
\section{用例と実践からの考察}
\else
  \section{Considerations from Examples and Practice}
\fi

\begin{enumerate}[resume, label=\textbf{Q\arabic*.}, leftmargin=2em]

  \item \label{qa:20250406a}
\ifJPN
  \textbf{即時文法は話しことばと同じですか?}
\else
  \textbf{Is Immediate Grammar the same as spoken language?}
\fi

\ifJPN
  \textbf{A.} 即時文法は「話しことば」と混同されがちですが、理論的には異なる概念です。即時文法は、心理的処理のタイミングや構造の柔軟性を基準とした文法モデルであり、口頭表現であっても調整文法が用いられることもあります。
\lse
  \textbf{A.} Immediate Grammar is often confused with "spoken language," but theoretically, they are different concepts. Immediate Grammar is a grammatical model based on the timing of psychological processing and structural flexibility, and even in oral expressions, adjusted grammar can be used.
\fi

  \item \label{qa:20250406b}
\ifJPN
  \textbf{Q\&A: なぜコーパスを用いた自動抽出を行わないのですか?}
\else
  \textbf{Q\&A: Why don't you use corpus-based automatic extraction?}
\fi

\ifJPN
  \textbf{A.} 近年の言語研究では、構文のパターンをコーパスから一括して抽出する方法が広く用いられています。これは形式的な構造(語順・係り受け・品詞パターン)に関する分析にとって非常に有効な手法です。
しかし、即時文法が対象とするのは、単なる構文のかたちではなく、「発話の仕方」「反応の仕方」「その瞬間に選択された表現」そのものであり、これは構文的なラベルや構造情報だけでは判断できません。 
たとえば、「そうそう、それそれ」といった発話や、「うん、でもさあ」のような連続表現は、文法的に曖昧で断片的に見えるかもしれませんが、即時的な判断・感情・反応の流れに基づく高度な構造を持っています。
こうした現象は、コーパス中にあっても、「検索条件に合致しない」「文法的なまとまりとして認識されない」といった理由で取りこぼされやすく、むしろ人間の目による文脈的判断と、経験に基づく記述的作業のほうが正確に取り扱えると考えています。
したがって、本研究では、既存のコーパスベースの方法と対立するのではなく、補完的なものとして、記述と言語直観に基づく観察的アプローチを採用しています。
\else
  \textbf{A.} In recent language research, methods for extracting syntactic patterns from corpora have been widely used. This is a very effective method for analyzing formal structures (word order, dependency, part-of-speech patterns).
However, what Immediate Grammar targets is not just the form of syntax but the "way of speaking," "way of responding," and "expressions chosen at that moment," which cannot be judged solely by syntactic labels or structural information.
For example, utterances like "そうそう、それそれ" (yes, yes, that's it) or continuous expressions like "うん、でもさあ" (yeah, but you know) may seem grammatically ambiguous and fragmented, but they have a sophisticated structure based on immediate judgments, emotions, and the flow of responses.
Such phenomena, even if present in the corpus, are often overlooked due to reasons like "not matching search criteria" or "not recognized as grammatical coherence," and it is believed that they can be more accurately handled through contextual judgment by human eyes and descriptive work based on experience.
Therefore, in this study, we adopt an observational approach based on description and linguistic intuition, not in opposition to existing corpus-based methods but as a complementary one.
\fi

\item \label{qa:20250405d}
\ifJPN
  \textbf{Q\&A: 即時文法は、会話分析やディスコース分析のような録音・文字化・分析の手続きを無視しているのですか?}
\else
  \textbf{Q\&A: Does Immediate Grammar ignore the procedures of recording, transcription, and analysis like conversation analysis or discourse analysis?}
\fi

\ifJPN
  \textbf{A.} いいえ、本研究における即時文法の立場は、ディスコース分析や会話分析が対象とする「発話の具体性」や「文脈への依存性」と共通する部分を多く持っています。特に、文脈に応じた語の選択、タイミング、相づち、言い直しなどの発話の連続性は、即時文法の重要な観察対象でもあります。
ただし、分析手法としては異なります。会話分析やディスコース分析では、音声の録音→文字化→発話単位の構造分析という手続きを通じて、参加者間の相互行為を記述します。これに対して、即時文法の目的は、こうした発話に見られるパターンを、発話者が瞬時に選択・操作している構造のモデルとして記述することです。
そのため、本研究では、録音・文字化という手続き自体は採用していませんが、それを前提にした発話資料や用例を十分に参照しており、観察の焦点が「使用されている即時的構造」にあるという点で、補完的なアプローチをとっています。
要するに、即時文法は、ディスコース分析や会話分析の成果と矛盾するものではなく、それらが描き出した「細かな発話現象」の背景にある構造的説明モデルとして機能することを目指しているのです。
\else
  \textbf{A.} No, the position of Immediate Grammar in this study shares many commonalities with the "specificity of utterances" and "contextual dependence" targeted by discourse analysis and conversation analysis. In particular, the continuity of utterances, such as word choice according to context, timing, backchanneling, and rephrasing, is also an important observation target of Immediate Grammar.
However, the analytical methods differ. In conversation analysis and discourse analysis, the procedure of recording audio → transcribing → structural analysis of utterance units is used to describe the interaction between participants. In contrast, the purpose of Immediate Grammar is to describe the patterns observed in such utterances as a model of the structures that speakers instantaneously select and manipulate.
Therefore, while this study does not adopt the procedures of recording and transcription, it sufficiently references utterance materials and examples based on them, taking a complementary approach in that the focus of observation is on the "immediate structures being used."
In short, Immediate Grammar does not contradict the results of discourse analysis or conversation analysis; rather, it aims to function as a structural explanatory model underlying the "detailed utterance phenomena" they depict.
\fi


\item \label{qa:20250413a}
\ifJPN
  \textbf{「継ぎ足し構文」や「連節文法」は、即時文法とどのような関係がありますか?}
\else
  \textbf{How are chaining constructions or chaining grammar related to Immediate Grammar?}
\fi

\ifJPN
  \textbf{A.} 「継ぎ足し構文」や「連節文法」は、複雑な文構造を動的かつ直感的に組み立てる際によく見られる形式であり、即時文法の重要な具体例の一つです。たとえば、話しながら順次内容を加えていく「〜て」「〜で」「〜けど」などの文は、文の全体像を事前に設計せずに生成される点で、即時文法の本質を示しています。このような文は、語りの流れに従って言語を構築していく人間の行動に密接に対応します。
\else
  \textbf{A.} Chaining constructions or chaining grammar represent a common form used when constructing complex sentences in a dynamic and intuitive way. They are a key manifestation of Immediate Grammar. For instance, sentences formed with expressions like “and then,” “but,” or “so,” which are added successively as one speaks, exemplify the nature of Immediate Grammar, as they are generated without pre-planning the entire sentence. These forms closely correspond to how people construct language in real-time.
\fi


\item \label{qa:20250413b}
\ifJPN
  \textbf{外国語話者にとって、「の」を使った修飾より「で」や「て」で継ぎ足す構文の方が自然に使えるのはなぜですか?}
\else
  \textbf{Why is it easier for non-native speakers to use constructions with "de" or "te" rather than complex noun phrases with "no"?}
\fi

\ifJPN
  \textbf{A.} 「の」を用いた修飾構文は、名詞の前に長い形容語句を積み重ねる必要があり、文全体を構造的に計画する調整文法的処理を求められます。これに対し、「で」や「て」などを使って行為や情報を順に継ぎ足していく構文は、出来事や行為の流れに沿って逐次的に発話できるため、即時文法的です。たとえば、「昨日駅で見た赤い帽子の女の子」は、「赤い帽子の女の子」が長く修飾された名詞であり、語順や係り受けが複雑になります。一方で、「昨日、駅に行って、赤い帽子をかぶった女の子を見た」という表現は、順を追って構成され、文としての完成を逐次先送りできるため、初級者にも自然で扱いやすい形式となります。これは英語でも同様で、``the girl who wore a red hat that I saw at the station yesterday'' よりも、``I went to the station yesterday and saw a girl. She was wearing a red hat.'' の方が習得が容易であることと対応しています。
\else
  \textbf{A.} Japanese constructions using ``no'' require stacking modifiers in front of a noun, which demands structural planning and is characteristic of Adjustive Grammar. In contrast, constructions using ``de'' or ``"te'' allow for sequential chaining of events or states, enabling spontaneous, step-by-step utterance—hallmarks of Immediate Grammar. For instance, the phrase "kinō eki de mita akai bōshi no onna no ko'' (``the girl in the red hat I saw at the station yesterday'') involves a complex noun phrase with multiple layers of modification. Meanwhile, a sentence like ``"Kinō eki ni itte, akai bōshi o kabutta onna no ko o mita" (``I went to the station yesterday and saw a girl wearing a red hat'') follows a chronological flow and allows the speaker to build the sentence incrementally. This is similar in English: ``The girl who wore a red hat that I saw at the station yesterday'' is structurally demanding, while ``I went to the station yesterday and saw a girl. She was wearing a red hat.'' is easier to produce and understand, especially for language learners.
\fi
\end{enumerate}

\ifJPN
\section{即時文法の実在性を支える先行研究の蓄積}
\else
\section{Accumulation of Previous Research Supporting the Existence of Immediate Grammar}
\fi
\label{sec:immediate_grammar_examples}

\ifJPN
\subsection{即時文法の実在性を支える観察の蓄積}
\else
\subsection{Accumulation of Observations Supporting the Existence of Immediate Grammar} 
\fi
\label{subsec:immediate_grammar_observations}

\ifJPN
本書では、「即時文法」という理論枠組みを提示しているが、その背景には、著者自身が行ってきた複数の実践的観察と記録がある。たとえば、『伊勢物語』や『土佐日記』の翻訳において、表現の自然な現代語訳を模索する過程で、従来の構文理論では説明が困難な即時的表現が多く見出された。また、aead(An expression a day)プロジェクトにおいては、日常的な日本語表現の中に即時文法の特徴を持つ表現を日々記録し、それらにタグと注釈を加えることで、言語使用の実態と即時文法の対応を浮かび上がらせてきた。
\else
In this book, we present the theoretical framework of ``Immediate Grammar,'' which is supported by multiple practical observations and records made by the author. For example, in the process of translating works like ``Ise Monogatari'' and ``Tosa Nikki,'' many immediate expressions were found that are difficult to explain using conventional syntactic theories. Additionally, in the aead (An expression a day) project, we have been recording expressions with characteristics of immediate grammar in everyday Japanese, tagging them, and adding annotations to highlight the correspondence between language use and immediate grammar.
\fi

\ifJPN
これらの作業は、直観的な主張ではなく、時間をかけて蓄積された観察の成果である。数百に及ぶ用例が、それぞれ即時文法の視点から解釈・記述されており、結果として、即時文法という理論が記述モデルとして有効であることを示す実証的な手がかりとなっている。
\else
These tasks are not based on intuitive claims but are the results of observations accumulated over time. Hundreds of examples have been interpreted and described from the perspective of immediate grammar, providing empirical clues that demonstrate the effectiveness of immediate grammar as a descriptive model.
\fi

\ifJPN
\subsection{従来の「話しことば性/書きことば性」概念とPGMの立場}
\else
\subsection{The Conventional Concepts of ``Spoken Language vs. Written Language'' and the Position of PGM}
\fi

\ifJPN
従来、言語スタイルの分類においては「話しことば(orality)」と「書きことば(literacy)」という二項対立、または連続体的なモデルが用いられてきた。たとえば \textcite{biber1988} は、複数のテキストタイプにおける構文的・語彙的特徴の出現傾向を統計的に分析し、話しことば的/書きことば的なスタイルが使用媒体を越えて存在することを示した。このように「話されている=話しことば」「書かれている=書きことば」という定義はすでに見直されており、より精緻なスタイルモデルが提案されている。

しかしながら、これらのモデルにおいても、話しことば性/書きことば性を判断する根拠が主に出現頻度や語彙傾向に依存しており、「なぜその形式が使われるのか」「その形式はどのように生成されたのか」という問いには十分に応えられていない。とくに、フィクション内のセリフや、スピーチ原稿のような事前準備された言語は、メディアと構文モードの不一致を引き起こす。

この点に対して、プロセス文法モデル(PGM)は、「話しことば性」「書きことば性」を使用媒体に基づく属性ではなく、\textbf{言語生成時の処理モード(即時性/調整性)}に基づく構文タイプと再定義する。すなわち、即時文法(Immediate Grammar)によって生成された表現は話しことば的であり、調整文法(Adjustive Grammar)に基づく表現は書きことば的である。この枠組みによって、話されていない「話しことば性」や、書かれていない「書きことば性」を理論的に記述することが可能となる。
\else
Traditionally, the classification of language styles has relied on a binary opposition or a continuum model between ``spoken language (orality)'' and ``written language (literacy).'' For example, \textcite{biber1988} statistically analyzed the syntactic and lexical features of multiple text types and demonstrated that spoken and written styles exist across different media. Thus, the definitions of ``spoken = spoken language'' and ``written = written language'' have already been revised, leading to the proposal of more refined style models.

However, even in these models, the criteria for determining spoken vs. written language primarily depend on frequency of occurrence and lexical tendencies, and they do not adequately address the questions of ``why that form is used'' or ``how that form is generated.'' In particular, pre-prepared language, such as dialogue in fiction or speech scripts, can lead to inconsistencies between media and syntactic modes.

In this regard, the Process Grammar Model (PGM) redefines ``spoken language'' and ``written language'' not as attributes based on the medium of use, but as syntactic types based on the processing modes during language generation (immediacy vs. adjustability). That is, expressions generated by Immediate Grammar are spoken-like, while those based on Adjustive Grammar are written-like. This framework allows for the theoretical description of ``spokenness'' that is not spoken and ``writtenness'' that is not written.
\fi

\ifJPN
\subsection{文法・表現・効果の三層構造}
\else
\subsection{The Three Layers: Grammar, Expression, and Effect}
\fi

\begin{figure}[htbp]
  \centering

  \resizebox{.7\textwidth}{!}{
  \begin{tikzpicture}[node distance=1.0cm, every node/.style={align=center}]
    % Grammar Layer
  \ifJPN
    \node (grammar) at (0,0) {\textbf{文法}};
    \node[left=1cm of grammar] (imm-gra) {即時文法};
    \node[right=1cm of grammar] (adj-gra) {調整文法};
  \else
    \node (grammar) at (0,0) {\textbf{Grammar}};
    \node[left=1cm of grammar] (imm-gra) {Immediate Grammar};
    \node[right=1cm of grammar] (adj-gra) {Adjustive Grammar};
  \fi
    
  % Expression Layer
  \ifJPN
    \node[below=of grammar] (expression) {\textbf{表現}};
    \node[below=of imm-gra] (imm-exp) {即時表現};
    \node[below=of adj-gra] (adj-exp) {調整表現};
  \else
    \node[below=of grammar] (expression) {\textbf{Expression}};
    \node[below=of imm-gra] (imm-exp) {Immediate Expression};
    \node[below=of adj-gra] (adj-exp) {Adjustive Expression};
  \fi
    
    % Effect Layer
  \ifJPN
    \node[below=of expression] (effect) {\textbf{効果}};
    \node[below=of imm-exp] (nat-eff) {自然な・生き生きした効果};
    \node[below=of adj-exp] (pol-eff) {丁寧で改まった効果};
  \else
    \node[below=of expression] (effect) {\textbf{Effect}};
    \node[below=of imm-exp] (nat-eff) {Natural and Spoken Effect};
    \node[below=of adj-exp] (pol-eff) {Polite and Public Effect};
  \fi
    
    % Arrows
    \draw[->] (imm-gra) -- (imm-exp);
    \draw[->] (adj-gra) -- (adj-exp);
    \draw[->] (imm-exp) -- (nat-eff);
    \draw[->] (adj-exp) -- (pol-eff);
    
    % Layer Labels
    \draw[dashed] (imm-gra) -- (grammar);
    \draw[dashed] (grammar) -- (adj-gra);
    \draw[dashed] (imm-exp) -- (expression);
    \draw[dashed] (expression) -- (adj-exp);
    \draw[dashed] (nat-eff) -- (effect);
    \draw[dashed] (effect) -- (pol-eff);
%    \node at (-5.5,0) {Grammar};
%    \node at (-5.5,-1.8) {Expression};
%    \node at (-5.5,-3.6) {Effect};
  \end{tikzpicture}
}

\ifJPN
  \caption{プロセス文法モデルにおける三層構造(交差なし)}
\else
  \caption{Three-layer structure in the Process Grammar Model (without intersection)}
\fi
\end{figure}


\ifJPN
プロセス文法モデルでは、言語使用をより精密に記述するために、「文法(Grammar)」「表現(Expression)」「効果(Effect)」という三層構造を採用する。この構造は、言語形式そのものではなく、生成プロセス・使用文脈・受け手の解釈効果に着目したものである。

\begin{itemize}
  \item \textbf{文法(Grammar)}:即時的に生成される文法(即時文法)と、調整・準備を経た文法(調整文法)
  \item \textbf{表現(Expression)}:実際に表出された形式(即時表現/調整表現)
  \item \textbf{効果(Effect)}:使用文脈に応じて、自然で生き生きした効果(Natural and Spoken Effect)や、丁寧で改まった効果(Polite and Public Effect)を生じる
\end{itemize}

この三層の交差によって、文法と使用文脈が一致しない場合でも、発話や記述の効果を理論的に説明することが可能となる。たとえば、即時文法によって生成された即時表現が、小説・映画・テレビドラマにおいて用いられた場合、自然で臨場感のある印象(Natural and Spoken Effect)を与える。一方、調整文法による表現がスピーチやナレーションで即時的に使用された場合には、丁寧で改まった印象(Polite and Public Effect)を与える。

このような文法と効果の交差を「襷掛け効果(Tasuki-gake Effect)」と呼ぶ。これは、即時性と調整性が相補的に用いられ、媒体やスタイルに依存しない柔軟な言語使用が可能であることを示している。
\else
To describe language use more precisely, the Process Grammar Model adopts a three-layer structure: ``Grammar,'' ``Expression,'' and ``Effect.'' This structure focuses not on surface features but on the generation process, contextual use, and the interpretive effect on the listener or reader.

\begin{itemize}
  \item \textbf{Grammar}: Immediate Grammar (real-time generation) vs. Adjustive Grammar (generated with preparation)
  \item \textbf{Expression}: The actual linguistic form expressed (Immediate vs. Adjustive Expression)
  \item \textbf{Effect}: Depending on the context, expressions evoke either a Natural and Spoken Effect or a Polite and Public Effect
\end{itemize}

This tripartite structure makes it possible to explain communicative effects even when the grammar and context do not align. For example, Immediate Expressions generated by Immediate Grammar, when used in novels or films, create a vivid and realistic impression (Natural and Spoken Effect). Conversely, Adjustive Expressions, when used in real-time speech such as public addresses, produce a refined and formal impression (Polite and Public Effect).

This cross-contextual usage is termed the ``Tasuki-gake Effect,'' highlighting how immediacy and adjustiveness can be flexibly combined beyond stylistic or media-based constraints.
\fi

\ifJPN
\subsection{Wallace Chafe の情報構造モデルとプロセス文法モデル(PGM)}
\else
\subsection{Wallace Chafe's Information Structure Model and the Process Grammar Model (PGM)}
\fi

\ifJPN
\textcite{chafe1982integration}は、話しことばと書きことばの差異を「即時性」と「調整性」の観点から理論的に整理し、以下の4つの概念軸を提示した:
\else
\textcite{chafe1982integration} theoretically organized the differences between spoken and written language from the perspectives of ``immediacy'' and ``adjustability,'' proposing the following four conceptual axes:
\fi

\begin{itemize}
\ifJPN
  \item \textbf{Idea Units(アイデアユニット)} \\ 発話は、意味・感情・リズムの単位として「思考の断片」が逐次生成されることで構成される。話しことばの即時性の基盤となる。
\else
  \item \textbf{Idea Units} \\ Speech is constructed by the sequential generation of ``fragments of thought'' as units of meaning, emotion, and rhythm. This forms the basis of the immediacy of spoken language.
\fi

\ifJPN
  \item \textbf{Integration(統合性)} \\ 情報をどれだけ圧縮し、構造的に統合するかの度合い。書きことばでは修飾・名詞化・複文構造によって統合性が高められる。
\else
\item \textbf{Integration} \\ The degree to which information is compressed and structurally integrated. In written language, integration is enhanced through modification, nominalization, and complex sentence structures.
\fi

\ifJPN
  \item \textbf{Involvement(関与性)} \\ 発話がどれだけ話者の感情や関係性に基づいて構築されているかを表す。関与性が高いほど、即時的で共感的な表現が増える。
\else
  \item \textbf{Involvement} \\ Indicates how much the speech is constructed based on the speaker's emotions and relationships. The higher the involvement, the more immediate and empathetic the expression becomes.
\fi

\ifJPN
  \item \textbf{Detachment(距離性)} \\ 抽象化・客観化された記述スタイル。語り手の姿勢が感情的関与から離れるとき、言語は距離性を帯び、調整文法に近づく。
\else
  \item \textbf{Detachment} \\ An abstracted and objective descriptive style. When the narrator's stance moves away from emotional involvement, the language takes on a detached quality, approaching adjustive grammar.
\fi

\end{itemize}

\begin{table}[htbp]\centering\small
\ifJPN
  \caption{Chafe のモデルと PGM の対照. 
  \textcite{chafe1982integration}を参考に作成}
\else
  \caption{Comparison of Chafe's Model and PGM.
    Adapted from \textcite{chafe1982integration}}
\fi
  \label{tab:chafe}
\ifJPN
  \begin{tabular}{lll} \noalign{\hrule height .8pt}
  Chafeの概念 & 即時文法(IG) & 調整文法(AG) \\ \hline
  Idea Units & 発話のスパート的な連鎖 & 統合された複文構造 \\
  Integration & 低(語順・構文の冗長性を含む) & 高(修飾の埋め込み・構造の密度) \\
  Involvement & 高(語り手の感情・共感の発露) & 低(客観・脱個人的視点) \\
  Detachment & 低(話者の具体的関与) & 高(抽象化・受動化・形式性) \\
\end{tabular}
\else
  \begin{tabular}{lp{60mm}p{60mm}} \noalign{\hrule height .8pt}
  Chafe's Concept & Immediate Grammar (IG) & Adjustive Grammar (AG) \\ \hline
  Idea Units & Spurt-like chain of speech & Integrated complex sentence structure \\
  Integration & Low (including redundancy in word order and syntax) & High (embedding of modifiers, density of structure) \\
  Involvement & High (emotional expression of the narrator, empathy) & Low (objectivity, depersonalized perspective) \\
  Detachment & Low (specific involvement of the speaker) & High (abstraction, passivation, formality) \\
\end{tabular}
\fi
\end{table}
\bigskip

\ifJPN
この対比は、PGM が定義する「即時性(Immediate Grammar)」と「調整性(Adjustive Grammar)」という文法使用の2軸を、Chafe の心理的・談話的次元における4概念——「アイデアユニット(思考の断片性)」「統合性(情報密度)」「関与性(感情的・対人的な結びつき)」「距離性(抽象性・脱個人性)」——によって、より多面的に位置づけるものである。
\else
This comparison positions the two axes of grammatical use defined by PGM—``immediacy'' and ``adjustability''—within Chafe's four concepts in psychological and discourse dimensions: ``Idea Units'' (fragmentation of thought), ``Integration'' (information density), ``Involvement'' (emotional and interpersonal connections), and ``Detachment'' (abstraction and depersonalization).
\fi

\ifJPN
表\ref{tab:chafe}が示すように、即時文法は断片的・感情的・対人的であり、話し手の主観的な関与が濃い言語使用を特徴とする。一方で調整文法は、構造が高度に統合され、抽象的かつ客観的な表現形式に向かう。Chafe の理論は、PGM の「連続体モデル(continuum model)」を裏付ける外部理論として重要な役割を果たし、両者のモデルの相補的関係を可視化するものである。
\else
As shown in Table \ref{tab:chafe}, immediate grammar is characterized by fragmented, emotional, and interpersonal language use, with a strong subjective involvement of the speaker. In contrast, adjustive grammar features highly integrated structures and moves toward abstract and objective forms of expression. Chafe's theory plays an important role as an external theory supporting PGM's ``continuum model,'' visualizing the complementary relationship between the two models.
\fi

\ifJPN
もっとも、Chafe のモデルにもいくつかの限界がある。
第一に、彼の「統合性」と「関与性」という二軸は心理的・談話的観点に重きを置いているため、言語構造そのものにおける形式的制約(たとえば語順、助詞、文の構造的許容性など)を十分に説明しきれない。
第二に、彼の「アイデアユニット」理論は話しことばの断片性を示すのに有効だが、語りの流れにおける構文的連鎖(たとえば呼応、反復、喚体句など)の扱いが限定的である。
\else
However, Chafe's model also has some limitations.
First, his two axes of ``integration'' and ``involvement'' place emphasis on psychological and discourse perspectives, which may not sufficiently explain formal constraints in the language structure itself (e.g., word order, particles, structural permissibility of sentences).
Second, while his theory of ``idea units'' is effective in demonstrating the fragmentation of spoken language, it has limited applicability in handling syntactic chains in narrative flow (e.g., anaphora, repetition, and vocative phrases).
\fi

\ifJPN
これに対して、PGM は即時文法と調整文法を文法形式と生成条件の観点から明示的に記述することで、Chafe の心理的次元に構造的次元を加える枠組みを提供している。
したがって、Chafe の理論は PGM の記述的深度を広げる参照枠として有効であり、逆に PGM は Chafe モデルの構造的・文法的限界を補完する形で位置づけることができる。
\else
In contrast, PGM explicitly describes immediate and adjustive grammar in terms of grammatical forms and generation conditions, providing a framework that adds structural dimensions to Chafe's psychological dimensions.
Thus, Chafe's theory serves as an effective reference framework for broadening the descriptive depth of PGM, while PGM can be positioned to complement the structural and grammatical limitations of Chafe's model.
\fi

\ifJPN
このように、Chafe が提示した心理的枠組みは、即時性と調整性の概念においてきわめて有効であるが、そのまま文法的記述に用いるには限定的である。PGM はこの理論的遺産を継承しつつ、言語の構造的現実と生成的文法の観点から、より精緻な統合を可能にしている。
\else
Thus, while Chafe's psychologically grounded model is highly effective in conceptualizing immediacy and adjustability, it remains limited when applied directly to grammatical description. The Process Grammar Model (PGM) inherits this theoretical legacy and extends it by integrating structural and generative dimensions of language.
\fi

\ifJPN
\subsection{Halliday の機能文法と生成理論の比較、そして PGM の位置づけ}
\else
\subsection{Comparison of Halliday's Functional Grammar and Generative Theory, and the Positioning of PGM}
\fi

\ifJPN
\subsubsection{話しことばと書きことばに対するHallidayの基本姿勢とSFLの必要性}
\else
\subsubsection{Halliday's Basic Attitude Toward Spoken and Written Language and the Necessity of SFL}
\fi

\ifJPN
Halliday(\textcite{halliday1994spoken})は、言語を単なる伝達の道具ではなく、「意味を創出する社会的資源」と捉える。この立場から、彼は特に話しことば(spoken language)と書きことば(written language)の違いに注目し、それぞれが果たす機能の相違を言語理論の中心に据えた。
\else
  Halliday (\textcite{halliday1994spoken}) views language not merely as a tool for communication but as a ``social resource for creating meaning.'' From this perspective, he particularly focuses on the differences between spoken language and written language, placing the functional distinctions of each at the center of linguistic theory.
\fi

\ifJPN
Halliday にとって、話しことばとは即時性と関与性に富んだ言語活動であり、文脈依存的で、リズムと即興性が支配する。一方、書きことばは高度に統合された構造をもち、抽象的で普遍化された意味表現が可能なモードである。
\else
Halliday considers spoken language to be a linguistic activity rich in immediacy and involvement, characterized by contextual dependency, rhythm, and improvisation. In contrast, written language possesses a highly integrated structure, allowing for abstract and generalized expressions of meaning.
\fi

\ifJPN
このような話しことばと書きことばの機能的差異を明示的に記述する必要から、Halliday は機能文法(Systemic Functional Linguistics, SFL)を展開した。SFL においては、言語は単に構文単位の集合ではなく、ideational(観念的)・interpersonal(対人的)・textual(テクスト的)という三つのメタ機能を同時に担うプロセスとして捉えられる。
\else
  This functional distinction between spoken and written language led Halliday to develop Systemic Functional Linguistics (SFL). In SFL, language is not merely a collection of syntactic units but is viewed as a process that simultaneously fulfills three meta-functions: ideational (representing ideas), interpersonal (interacting with others), and textual (organizing text).
\fi

\ifJPN
とりわけ、Halliday は書きことばを「文法的に高度に統合された形式(grammatically intricate)」、話しことばを「構造的には単純だが意味的に豊かな形式(lexically dense)」と位置づけ、従来の構文中心のモデルでは捉えきれない「使用の場に応じた意味機能の違い」を理論化した。これこそが、彼が構築した選択体系機能言語学(SFL)の出発点であり、社会的状況に即した言語の選択可能性を記述するための基盤となっている。
\else
In particular, Halliday characterizes written language as ``grammatically intricate'' and spoken language as ``lexically dense but structurally simple,'' theorizing the differences in meaning functions that cannot be captured by traditional syntax-centered models. This serves as the starting point for his development of Systemic Functional Linguistics (SFL), providing a foundation for describing the choices of language that are contextually relevant to social situations.
\fi

\ifJPN
\subsubsection{理論構築の目的の違い}
\else
\subsubsection{Differences in the Purpose of Theoretical Construction}
\fi

\ifJPN
Halliday の機能文法(SFL)と Chomsky の生成理論(\textcite{chomsky1965aspects})は、言語に対する基本的な捉え方と理論構築の目的において、根本的に異なっている。
\else
  Halliday's Systemic Functional Linguistics (SFL) and Chomsky's Generative Theory (\textcite{chomsky1965aspects}) fundamentally differ in their basic understanding of language and the purpose of theoretical construction.
\fi
\ifJPN
生成理論は、言語を人間に生得的に備わっている能力と捉え、その内部構造や構文規則を数理的に明示化することを目指している。ここでは、理想化された話者・聞き手という抽象的存在を前提とし、主に NP や VP といった構文単位に基づいて文を分析する。証明においては、経済性や最小限の規則からなる普遍文法(UG)を構築することが重視される。
\else
Generative theory views language as an innate ability of humans, aiming to mathematically clarify its internal structure and syntactic rules. In this framework, an idealized speaker-listener is assumed, and sentences are primarily analyzed based on syntactic units such as NP (noun phrase) and VP (verb phrase). The focus is on constructing a universal grammar (UG) based on principles of economy and minimalism.
\fi
\ifJPN
一方で、Halliday の機能文法は、言語を社会的行為の資源と見なし、実際の文脈において使われる言語の意味機能を明らかにしようとする。ここでは、理想話者ではなく、現実の書き手・話し手が主語となり、文の単位も句ではなく節(clause)が中心となる。意味の使用実態に根ざしており、構文の妥当性ではなく、文脈における機能的妥当性が重視される。
\else
In contrast, Halliday's functional grammar views language as a resource for social action, aiming to clarify the meaning functions of language used in actual contexts. Here, the focus is not on an ideal speaker but on real writers and speakers, with clauses (not phrases) being the central unit of analysis. It is rooted in the actual use of meaning, emphasizing functional validity in context rather than syntactic correctness.
\fi

\ifJPN
このように、Chomsky は文法の「内部構造」を、Halliday は文法の「社会的機能」を記述の主眼としており、それぞれの理論は異なる問いに答えようとしている。
\else
Thus, Chomsky focuses on the ``internal structure'' of grammar, while Halliday emphasizes the ``social function'' of grammar, with each theory attempting to address different questions.
\fi
\ifJPN
それぞれの理論は、異なる次元の言語的問いに取り組んでおり、どちらも独自の意義を持つ。生成理論は、言語能力の普遍的構造とその数理的記述において深い洞察を与えており、文法理論の形式的精緻化に大きな貢献をしてきた。一方、SFLは、社会的場面での言語使用の多様性を反映し、文法を意味生成の手段として捉えるという点で、教育・言語習得・メディア研究など実践的領域において重要な理論的基盤を提供している。
\else
Each theory addresses different dimensions of linguistic questions and holds its own significance. Generative theory provides deep insights into the universal structure of language ability and its mathematical description, contributing significantly to the formal refinement of grammatical theory. On the other hand, SFL reflects the diversity of language use in social contexts and offers an important theoretical foundation in practical fields such as education, language acquisition, and media studies by viewing grammar as a means of meaning generation.
\fi

\ifJPN
\subsubsection{Halliday の強みと生成理論との相違点}
\else
\subsubsection{Halliday's Strengths and Differences from Generative Theory}
\fi

\ifJPN
Halliday の機能文法は、いくつかの点において生成理論とは異なる強みを持っている。第一に、Halliday は言語を使用の観点からとらえ、命令文や願望文、評価表現など、日常的な文法使用を理論の出発点とする。第二に、文の構成を主語や述語といった構文的枠組みではなく、行為・参加者・状況という意味的単位で記述する点が特徴的である。第三に、言語使用が常に社会的文脈の中で起こることを前提とし、文体(レジスター)やジャンルなど社会的条件と不可分のものとして扱う。第四に、一つの文が三つのメタ機能――ideational(観念的)、interpersonal(対人的)、textual(テクスト的)――を同時に実現するという点に注目する。この重層的な構造こそが、SFLが理論的に最も特徴的であり、他の文法理論では十分に扱われてこなかった多機能性の統合を示している。
\else
Halliday's functional grammar possesses several strengths that differ from generative theory. First, Halliday approaches language from the perspective of use, making everyday grammatical usage—such as imperatives, desideratives, and evaluative expressions—the starting point of his theory. Second, he describes the structure of sentences not in terms of syntactic frameworks like subjects and predicates but in terms of semantic units such as actions, participants, and situations. Third, he assumes that language use always occurs within a social context, treating it as inseparable from social conditions such as style (register) and genre. Fourth, he emphasizes that a single sentence can simultaneously realize three meta-functions: ideational (representing ideas), interpersonal (interacting with others), and textual (organizing text). This layered structure is what makes SFL theoretically distinctive and demonstrates the integration of multifunctionality that has not been adequately addressed by other grammatical theories.
\fi

\ifJPN
SFL に対しては、生成理論の側からいくつかの批判的見解が提示されている\parencite{newmeyer1998language,pullum2010central}。まず第一に、SFL の規則体系には形式的に厳密な文法生成規則が明示されておらず、この点が「形式的定義の欠如」として批判の対象となる。第二に、SFL における構文と意味の関係は密接であるがゆえに、生成理論が採る「構文と意味の峻別」という原則とは相容れず、その区別が曖昧であると見なされる。第三に、SFL の構造記述は数理的整合性や再現可能性の面で不十分とされ、形式文法としての精緻なモデル化が困難であるという指摘がなされている。
\else
Critiques from the generative side have been directed at SFL \parencite{newmeyer1998language,pullum2010central}. First, the system of rules in SFL lacks formally rigorous grammatical generation rules, which is criticized as a ``lack of formal definition.'' Second, while the relationship between syntax and meaning in SFL is close, it is seen as incompatible with the generative principle of ``distinguishing syntax from semantics,'' leading to perceptions of ambiguity. Third, SFL's structural descriptions are considered insufficient in terms of mathematical consistency and reproducibility, making it difficult to model them as formal grammars.
\fi

\ifJPN
このように、SFL の柔軟で意味重視の枠組みと生成理論の求める明確性と形式的厳格さは言語の事実を詳らかにする上で重要で両者ともに重要であるが、統合的枠組みにより、これらの良さを生かすべきであると考える。
\else
This way, the flexible and meaning-oriented framework of SFL and the clarity and formal rigor sought by generative theory are both important for elucidating linguistic facts, and it is believed that an integrated framework should leverage these strengths.
\fi

\ifJPN
\subsubsection{PGMが文法の記述の点で、社会的機能の記述に貢献できること}
\else
\subsubsection{PGM's Contribution to the Description of Social Functions in Grammar}
\fi

\ifJPN
PGM(プロセス文法モデル)は、SFLが重視してきた社会的機能の観点を保持しながらも、それをより明示的に文法構造に結びつけるための補助的な枠組みとして機能する。特に、PGMは言語表現が出現する「使用条件」を重視し、発話が即時的に生成されるか、それともあらかじめ調整された構文に基づくかという時間的特性に注目する。この「即時性/調整性」という軸は、Halliday の対人的機能やテクスト形成的機能と深く結びついており、実際の言語使用の場における構文的選択の背後にある要因を記述可能とする。
\else
The Process Grammar Model (PGM) functions as a supplementary framework that retains the social function perspective emphasized by SFL while making it more explicitly connected to grammatical structures. In particular, PGM focuses on the ``conditions of use'' under which linguistic expressions emerge, paying attention to whether speech is generated immediately or based on pre-adjusted syntax. This axis of ``immediacy/adjustability'' is deeply connected to Halliday's interpersonal and textual functions, allowing for the description of the factors behind syntactic choices in actual language use.
\fi

\ifJPN
また、PGMが導入する「効果層(Effect Layer)」という概念は、単に構文形式が意味を担うだけではなく、それがどのような語用論的効果をもたらすか(たとえば親しさ、敬意、強調、即興性など)を記述するための有効な手段である。これは、Hallidayが提唱する文法の対人的機能と部分的に重なりながらも、PGMにおいてはその効果が発話生成のプロセスの中で動的に生起するものとして扱われる点において、機能文法に対する拡張的補足を与えている。
\else
Additionally, the concept of the ``Effect Layer'' introduced by PGM serves as an effective means to describe not only how syntactic forms carry meaning but also what pragmatic effects they produce (e.g., familiarity, respect, emphasis, immediacy). While this partially overlaps with Halliday's interpersonal function of grammar, PGM treats these effects as dynamically arising within the process of speech generation, providing an extended supplement to functional grammar.
\fi

\ifJPN
このように、PGMは構文と機能の対応を静的な体系ではなく、生成の過程そのものにおける可変的・選択的な現象として捉え直すことにより、SFLが描き出してきた社会的意味のネットワークを、より実行可能な文法記述のレベルに接続する可能性を持っている。
\else
Thus, PGM has the potential to reconceptualize the correspondence between syntax and function not as a static system but as a variable and selective phenomenon within the process of generation, thereby connecting the network of social meanings depicted by SFL to a more actionable level of grammatical description.
\fi




%今後の展開として:
%    🧭 Chafe・Halliday・Chomsky を並置する導入節の整備
%    📘 各理論の用語集的まとめ(附録形式)
%    🌐 英文展開や要約パラグラフの整備
%    🧪 実例(会話・文・詩・教育場面など)への応用節

% grammar-as-performance.tex

\ifJPN
\section{文法と演出:語りの構造と即時文法}
\else
\section{Grammar and Performance: The Structure of Narration and Immediate Grammar}
\fi

\ifJPN
従来の言語学において、語りの「演出効果」が文法構造によって直接的に支えられているという視点は、必ずしも体系的に扱われてこなかった。生成文法は意味と構造に、語用論は話し手の意図に焦点を当ててきたが、語りにおける構文の選択が、即時的な演出(surprise, reveal, buildup)の効果をどのように生成しうるかについては、部分的な言及にとどまっている。
\else
In traditional linguistics, the perspective that the "performance effect" of narration is directly supported by grammatical structure has not been systematically addressed. Generative grammar has focused on meaning and structure, while pragmatics has concentrated on the speaker's intention. However, there has been only partial mention of how the choice of syntax in narration can generate immediate performance effects (surprise, reveal, buildup).
\fi

\ifJPN
プロセス文法モデル(PGM)では、文法は単なる意味構築の枠組みにとどまらず、語り手が聞き手の注意・感情・予測を制御する「演出の構造」として機能することを前提とする。特に即時文法においては、語り手がその場で思い浮かべた順序で情報を提示する際に、主語の遅延や場所句の先行、動作句の先行挿入などがしばしば現れる。これらの構造は、認知的処理における「舞台設定→drum-roll→焦点開示」という演出的効果を持ち、構文そのものがその効果を支えている。
\else
In the Process Grammar Model (PGM), grammar is not merely a framework for meaning construction but functions as a "structure of performance" that allows the narrator to control the listener's attention, emotions, and predictions. In particular, in immediate grammar, when the narrator presents information in the order they think of it on the spot, structures such as subject delay, preposing of locative phrases, and insertion of action phrases often occur. These structures have a performative effect of "setting the stage → drum-roll → focus disclosure" in cognitive processing, and the syntax itself supports that effect.
\fi

\ifJPN
たとえば、「山の中から、出てきた出てきた、羊さんたちでーす」という語りは、「場所→動詞句→主語」という語順であり、英語の Locative Inversion(From the box came a bird.)と構造的に一致する。しかしこの語順は、即時的に生成された語りの流れであり、調整された表現の即時使用による襷掛け効果ではなく、即時文法の内在的構文である。
\else
For example, the narration "From the mountains, here come the sheep!" follows the order "locative → verb phrase → subject," structurally matching English Locative Inversion (From the box came a bird.). However, this word order is part of an immediately generated narrative flow and is not an effect of immediate use of adjusted expressions; it is an intrinsic syntax of immediate grammar.
\fi

\ifJPN
このような例においては、演出は文法に従属するのではなく、むしろ文法が演出を可能にする枠組みそのものとなっている。よって、PGMにおける「文法と演出」の関係は、構文選択の動機が「意味」だけでなく「効果(エフェクト)」にあることを明示する必要がある。
\else
In such examples, the performance does not depend on grammar; rather, grammar itself becomes the framework that enables the performance. Therefore, in PGM, the relationship between "grammar and performance" needs to clarify that the motivation for syntactic choice lies not only in "meaning" but also in "effect."
\fi


\if0
\ifJPN
\subsection*{今後の課題}
\else
\subsection*{Future Tasks}
\fi

\begin{itemize}
  \ifJPN
  \item AEAD における即時表現のうち、「焦点遅延」や「出現語順(場所→動作→主語)」が用いられている事例の収集と分類。
  \item 『伊勢物語』『土佐日記』の語りにおける即時的演出構造の記述と、和文構文との対応の明確化。
  \item BibLaTeX を用いた文献注の構築。以下の文献は本節に関連する基本資料である。

  \else

  \item Collection and classification of instances in AEAD where immediate expressions such as "focus delay" and "emergent word order (locative → action → subject)" are used.
  \item Description of immediate performance structures in the narration of "Ise Monogatari" and "Tosa Nikki," and clarification of their correspondence with Japanese syntax.
  \item Construction of bibliographic notes using BibLaTeX. The following references are basic materials related to this section:

\fi

  \begin{itemize}
      \item Birner, B. J. (1996). \textit{The Discourse Function of Inversion in English}. Garland Publishing.
      \item Huddleston, R., \& Pullum, G. K. (2002). \textit{The Cambridge Grammar of the English Language}. Cambridge University Press.
      \item Ward, G., \& Birner, B. J. (1995). Definiteness and the English Existential Construction. \textit{Language}, 71(4), 722–742.
      \item Quirk, R. et al. (1985). \textit{A Comprehensive Grammar of the English Language}. Longman.
    \end{itemize}
\end{itemize}
\fi


\printbibliography
\end{document}
