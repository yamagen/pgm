\ifJPN
\section{Q\&A セクションについて}
\else
\section{About the Q\&A Section}
\fi
\label{sec:QandA}

\ifJPN
本書では、即時文法の理論モデルを提示し、それに基づく実例や応用を通じて新たな言語理解の枠組みを探ってきた。しかし、こうした新しい理論を提示するにあたっては、読者の視点からさまざまな疑問や確認したい点が生まれることも予想される。
\else
In this book, we have presented the theoretical model of immediate grammar and explored a new framework for language understanding through examples and applications based on it. However, when introducing such a new theory, it is expected that various questions and points of confirmation will arise from the reader's perspective.
\fi

\ifJPN
この「Q\&A セクション」では、そうした疑問に対する著者の立場や補足的な考察を記録しておくことを目的とする。また、このセクションには、翻訳作業(たとえば『土佐日記』『伊勢物語』)や、日々の即時文法表現の収集作業(aeadプロジェクト)において、思考の過程で浮かび上がった疑問や発見も含めていく。
\else
This ``Q\&A Section'' aims to record the author's position and supplementary considerations regarding such questions. Additionally, this section will include questions and discoveries that have emerged during the translation work (for example, ``Tosa Nikki'' and ``Ise Monogatari'') and the daily collection of immediate grammar expressions (aead project).
\fi

\ifJPN
本文に組み込むには構成上の調整が必要となるが、それとは別に「書くたびに微妙なズレが生じる」ことを避け、思考のまとまりを維持するために、このセクションを用いる。必要に応じて本文との連携を後から再検討することを想定しており、本セクションは柔軟な蓄積と検討の場として位置づけられている。
\else
To avoid ``subtle discrepancies that arise every time I write'' and to maintain the coherence of thought, this section will be used. It is assumed that the integration with the main text will be reconsidered later as needed, and this section is positioned as a flexible space for accumulation and examination.
\fi

\ifJPN
\subsection{Q\&A: 即時文法とその理論的背景}
\else
\subsection{Q\&A: Immediate Grammar and Its Theoretical Background}
\fi

% 連番Q&A:最初のグループ
\begin{enumerate}[label=\textbf{Q\arabic*.}, leftmargin=2em]

  \item \label{qa:20250405a}
\ifJPN
  \textbf{即時文法とはどのような理論ですか? また、既存の構文理論とどのように位置づけられるのですか?}
\else
  \textbf{What kind of theory is Immediate Grammar? How is it positioned in relation to existing syntactic theories?}
\fi

\ifJPN
  \textbf{A.} 即時文法は、言語使用に見られる即時的かつ柔軟な表現形式に注目し、それらを理論的に説明しうる構造として提案されるものです。このモデルは、句構造文法のような構文中心の理論と対置されるものではなく、同等の立場において、人間の言語行動を異なる観点から記述しようとする仮説的枠組みです。
\else
  \textbf{A.} Immediate Grammar focuses on the immediate and flexible forms of expression observed in language use and proposes them as structures that can be theoretically explained. This model is not positioned in opposition to syntactic-centered theories like phrase structure grammar but is a hypothetical framework that aims to describe human language behavior from a different perspective.
\fi

  \item \label{qa:20250405b}
\ifJPN
  \textbf{即時文法は、どのような点で言語記述に貢献するのですか?}
\else
  \textbf{In what ways does Immediate Grammar contribute to language description?}
\fi

\ifJPN
  \textbf{A.} 即時文法という視点を導入することで、これまで「例外的」または「構造に乏しい」とされてきた発話が、体系的な言語使用の一形態として位置づけられるようになります。これは、即時文法がそうした発話に対して理論的な「説明のスロット」を提供するモデルであることの証とも言えます。
\else
  \textbf{A.} By introducing the perspective of Immediate Grammar, utterances that have previously been considered "exceptional" or "lacking structure" can be positioned as a systematic form of language use. This can be seen as evidence that Immediate Grammar provides a theoretical "slot for explanation" for such utterances.
\fi

\end{enumerate}

\ifJPN
\section{用例と実践からの考察}
\else
  \section{Considerations from Examples and Practice}
\fi

\begin{enumerate}[resume, label=\textbf{Q\arabic*.}, leftmargin=2em]

  \item \label{qa:20250406a}
\ifJPN
  \textbf{即時文法は話しことばと同じですか?}
\else
  \textbf{Is Immediate Grammar the same as spoken language?}
\fi

\ifJPN
  \textbf{A.} 即時文法は「話しことば」と混同されがちですが、理論的には異なる概念です。即時文法は、心理的処理のタイミングや構造の柔軟性を基準とした文法モデルであり、口頭表現であっても調整文法が用いられることもあります。
\else
  \textbf{A.} Immediate Grammar is often confused with "spoken language," but theoretically, they are different concepts. Immediate Grammar is a grammatical model based on the timing of psychological processing and structural flexibility, and even in oral expressions, adjusted grammar can be used.
\fi

  \item \label{qa:20250406b}
\ifJPN
  \textbf{Q\&A: なぜコーパスを用いた自動抽出を行わないのですか?}
\else
  \textbf{Q\&A: Why don't you use corpus-based automatic extraction?}
\fi

\ifJPN
  \textbf{A.} 近年の言語研究では、構文のパターンをコーパスから一括して抽出する方法が広く用いられています。これは形式的な構造(語順・係り受け・品詞パターン)に関する分析にとって非常に有効な手法です。
しかし、即時文法が対象とするのは、単なる構文のかたちではなく、「発話の仕方」「反応の仕方」「その瞬間に選択された表現」そのものであり、これは構文的なラベルや構造情報だけでは判断できません。 
たとえば、「そうそう、それそれ」といった発話や、「うん、でもさあ」のような連続表現は、文法的に曖昧で断片的に見えるかもしれませんが、即時的な判断・感情・反応の流れに基づく高度な構造を持っています。
こうした現象は、コーパス中にあっても、「検索条件に合致しない」「文法的なまとまりとして認識されない」といった理由で取りこぼされやすく、むしろ人間の目による文脈的判断と、経験に基づく記述的作業のほうが正確に取り扱えると考えています。
したがって、本研究では、既存のコーパスベースの方法と対立するのではなく、補完的なものとして、記述と言語直観に基づく観察的アプローチを採用しています。
\else
  \textbf{A.} In recent language research, methods for extracting syntactic patterns from corpora have been widely used. This is a very effective method for analyzing formal structures (word order, dependency, part-of-speech patterns).
However, what Immediate Grammar targets is not just the form of syntax but the "way of speaking," "way of responding," and "expressions chosen at that moment," which cannot be judged solely by syntactic labels or structural information.
For example, utterances like "そうそう、それそれ" (yes, yes, that's it) or continuous expressions like "うん、でもさあ" (yeah, but you know) may seem grammatically ambiguous and fragmented, but they have a sophisticated structure based on immediate judgments, emotions, and the flow of responses.
Such phenomena, even if present in the corpus, are often overlooked due to reasons like "not matching search criteria" or "not recognized as grammatical coherence," and it is believed that they can be more accurately handled through contextual judgment by human eyes and descriptive work based on experience.
Therefore, in this study, we adopt an observational approach based on description and linguistic intuition, not in opposition to existing corpus-based methods but as a complementary one.
\fi

\item \label{qa:20250405d}
\ifJPN
  \textbf{Q\&A: 即時文法は、会話分析やディスコース分析のような録音・文字化・分析の手続きを無視しているのですか?}
\else
  \textbf{Q\&A: Does Immediate Grammar ignore the procedures of recording, transcription, and analysis like conversation analysis or discourse analysis?}
\fi

\ifJPN
  \textbf{A.} いいえ、本研究における即時文法の立場は、ディスコース分析や会話分析が対象とする「発話の具体性」や「文脈への依存性」と共通する部分を多く持っています。特に、文脈に応じた語の選択、タイミング、相づち、言い直しなどの発話の連続性は、即時文法の重要な観察対象でもあります。
ただし、分析手法としては異なります。会話分析やディスコース分析では、音声の録音→文字化→発話単位の構造分析という手続きを通じて、参加者間の相互行為を記述します。これに対して、即時文法の目的は、こうした発話に見られるパターンを、発話者が瞬時に選択・操作している構造のモデルとして記述することです。
そのため、本研究では、録音・文字化という手続き自体は採用していませんが、それを前提にした発話資料や用例を十分に参照しており、観察の焦点が「使用されている即時的構造」にあるという点で、補完的なアプローチをとっています。
要するに、即時文法は、ディスコース分析や会話分析の成果と矛盾するものではなく、それらが描き出した「細かな発話現象」の背景にある構造的説明モデルとして機能することを目指しているのです。
\else
  \textbf{A.} No, the position of Immediate Grammar in this study shares many commonalities with the "specificity of utterances" and "contextual dependence" targeted by discourse analysis and conversation analysis. In particular, the continuity of utterances, such as word choice according to context, timing, backchanneling, and rephrasing, is also an important observation target of Immediate Grammar.
However, the analytical methods differ. In conversation analysis and discourse analysis, the procedure of recording audio → transcribing → structural analysis of utterance units is used to describe the interaction between participants. In contrast, the purpose of Immediate Grammar is to describe the patterns observed in such utterances as a model of the structures that speakers instantaneously select and manipulate.
Therefore, while this study does not adopt the procedures of recording and transcription, it sufficiently references utterance materials and examples based on them, taking a complementary approach in that the focus of observation is on the "immediate structures being used."
In short, Immediate Grammar does not contradict the results of discourse analysis or conversation analysis; rather, it aims to function as a structural explanatory model underlying the "detailed utterance phenomena" they depict.
\fi


\item \label{qa:20250413a}
\ifJPN
  \textbf{「継ぎ足し構文」や「連節文法」は、即時文法とどのような関係がありますか?}
\else
  \textbf{How are chaining constructions or chaining grammar related to Immediate Grammar?}
\fi

\ifJPN
  \textbf{A.} 「継ぎ足し構文」や「連節文法」は、複雑な文構造を動的かつ直感的に組み立てる際によく見られる形式であり、即時文法の重要な具体例の一つです。たとえば、話しながら順次内容を加えていく「〜て」「〜で」「〜けど」などの文は、文の全体像を事前に設計せずに生成される点で、即時文法の本質を示しています。このような文は、語りの流れに従って言語を構築していく人間の行動に密接に対応します。
\else
  \textbf{A.} Chaining constructions or chaining grammar represent a common form used when constructing complex sentences in a dynamic and intuitive way. They are a key manifestation of Immediate Grammar. For instance, sentences formed with expressions like “and then,” “but,” or “so,” which are added successively as one speaks, exemplify the nature of Immediate Grammar, as they are generated without pre-planning the entire sentence. These forms closely correspond to how people construct language in real-time.
\fi


\item \label{qa:20250413b}
\ifJPN
  \textbf{外国語話者にとって、「の」を使った修飾より「で」や「て」で継ぎ足す構文の方が自然に使えるのはなぜですか?}
\else
  \textbf{Why is it easier for non-native speakers to use constructions with "de" or "te" rather than complex noun phrases with "no"?}
\fi

\ifJPN
  \textbf{A.} 「の」を用いた修飾構文は、名詞の前に長い形容語句を積み重ねる必要があり、文全体を構造的に計画する調整文法的処理を求められます。これに対し、「で」や「て」などを使って行為や情報を順に継ぎ足していく構文は、出来事や行為の流れに沿って逐次的に発話できるため、即時文法的です。たとえば、「昨日駅で見た赤い帽子の女の子」は、「赤い帽子の女の子」が長く修飾された名詞であり、語順や係り受けが複雑になります。一方で、「昨日、駅に行って、赤い帽子をかぶった女の子を見た」という表現は、順を追って構成され、文としての完成を逐次先送りできるため、初級者にも自然で扱いやすい形式となります。これは英語でも同様で、``the girl who wore a red hat that I saw at the station yesterday'' よりも、``I went to the station yesterday and saw a girl. She was wearing a red hat.'' の方が習得が容易であることと対応しています。
\else
  \textbf{A.} Japanese constructions using ``no'' require stacking modifiers in front of a noun, which demands structural planning and is characteristic of Adjustive Grammar. In contrast, constructions using ``de'' or ``"te'' allow for sequential chaining of events or states, enabling spontaneous, step-by-step utterance—hallmarks of Immediate Grammar. For instance, the phrase "kinō eki de mita akai bōshi no onna no ko'' (``the girl in the red hat I saw at the station yesterday'') involves a complex noun phrase with multiple layers of modification. Meanwhile, a sentence like ``"Kinō eki ni itte, akai bōshi o kabutta onna no ko o mita" (``I went to the station yesterday and saw a girl wearing a red hat'') follows a chronological flow and allows the speaker to build the sentence incrementally. This is similar in English: ``The girl who wore a red hat that I saw at the station yesterday'' is structurally demanding, while ``I went to the station yesterday and saw a girl. She was wearing a red hat.'' is easier to produce and understand, especially for language learners.
\fi
\end{enumerate}
