% grammar-as-performance.tex

\ifJPN
\section{文法と演出:語りの構造と即時文法}
\else
\section{Grammar and Performance: The Structure of Narration and Immediate Grammar}
\fi

\ifJPN
従来の言語学において、語りの「演出効果」が文法構造によって直接的に支えられているという視点は、必ずしも体系的に扱われてこなかった。生成文法は意味と構造に、語用論は話し手の意図に焦点を当ててきたが、語りにおける構文の選択が、即時的な演出(surprise, reveal, buildup)の効果をどのように生成しうるかについては、部分的な言及にとどまっている。
\else
In traditional linguistics, the perspective that the "performance effect" of narration is directly supported by grammatical structure has not been systematically addressed. Generative grammar has focused on meaning and structure, while pragmatics has concentrated on the speaker's intention. However, there has been only partial mention of how the choice of syntax in narration can generate immediate performance effects (surprise, reveal, buildup).
\fi

\ifJPN
プロセス文法モデル(PGM)では、文法は単なる意味構築の枠組みにとどまらず、語り手が聞き手の注意・感情・予測を制御する「演出の構造」として機能することを前提とする。特に即時文法においては、語り手がその場で思い浮かべた順序で情報を提示する際に、主語の遅延や場所句の先行、動作句の先行挿入などがしばしば現れる。これらの構造は、認知的処理における「舞台設定→drum-roll→焦点開示」という演出的効果を持ち、構文そのものがその効果を支えている。
\else
In the Process Grammar Model (PGM), grammar is not merely a framework for meaning construction but functions as a "structure of performance" that allows the narrator to control the listener's attention, emotions, and predictions. In particular, in immediate grammar, when the narrator presents information in the order they think of it on the spot, structures such as subject delay, preposing of locative phrases, and insertion of action phrases often occur. These structures have a performative effect of "setting the stage → drum-roll → focus disclosure" in cognitive processing, and the syntax itself supports that effect.
\fi

\ifJPN
たとえば、「山の中から、出てきた出てきた、羊さんたちでーす」という語りは、「場所→動詞句→主語」という語順であり、英語の Locative Inversion(From the box came a bird.)と構造的に一致する。しかしこの語順は、即時的に生成された語りの流れであり、調整された表現の即時使用による襷掛け効果ではなく、即時文法の内在的構文である。
\else
For example, the narration "From the mountains, here come the sheep!" follows the order "locative → verb phrase → subject," structurally matching English Locative Inversion (From the box came a bird.). However, this word order is part of an immediately generated narrative flow and is not an effect of immediate use of adjusted expressions; it is an intrinsic syntax of immediate grammar.
\fi

\ifJPN
このような例においては、演出は文法に従属するのではなく、むしろ文法が演出を可能にする枠組みそのものとなっている。よって、PGMにおける「文法と演出」の関係は、構文選択の動機が「意味」だけでなく「効果(エフェクト)」にあることを明示する必要がある。
\else
In such examples, the performance does not depend on grammar; rather, grammar itself becomes the framework that enables the performance. Therefore, in PGM, the relationship between "grammar and performance" needs to clarify that the motivation for syntactic choice lies not only in "meaning" but also in "effect."
\fi


\if0
\ifJPN
\subsection*{今後の課題}
\else
\subsection*{Future Tasks}
\fi

\begin{itemize}
  \ifJPN
  \item AEAD における即時表現のうち、「焦点遅延」や「出現語順(場所→動作→主語)」が用いられている事例の収集と分類。
  \item 『伊勢物語』『土佐日記』の語りにおける即時的演出構造の記述と、和文構文との対応の明確化。
  \item BibLaTeX を用いた文献注の構築。以下の文献は本節に関連する基本資料である。

  \else

  \item Collection and classification of instances in AEAD where immediate expressions such as "focus delay" and "emergent word order (locative → action → subject)" are used.
  \item Description of immediate performance structures in the narration of "Ise Monogatari" and "Tosa Nikki," and clarification of their correspondence with Japanese syntax.
  \item Construction of bibliographic notes using BibLaTeX. The following references are basic materials related to this section:

\fi

  \begin{itemize}
      \item Birner, B. J. (1996). \textit{The Discourse Function of Inversion in English}. Garland Publishing.
      \item Huddleston, R., \& Pullum, G. K. (2002). \textit{The Cambridge Grammar of the English Language}. Cambridge University Press.
      \item Ward, G., \& Birner, B. J. (1995). Definiteness and the English Existential Construction. \textit{Language}, 71(4), 722–742.
      \item Quirk, R. et al. (1985). \textit{A Comprehensive Grammar of the English Language}. Longman.
    \end{itemize}
\end{itemize}
\fi
