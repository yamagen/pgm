
\newcommand{\TasukiGakeEffectJa}{
\textbf{襷掛け効果(tasuki-gake effect)}とは、「即時文法」と「調整文法」が本来の場面を越えて交差的に使用されることによって生じる文体的効果を指す。
即時文法が調整文法的媒体(例:小説、スピーチ)に用いられると、発話に速度や臨場感が加わる。
逆に、調整文法が即時の発話場面で用いられると、丁寧さや形式性が帯びる。
このような交差的な使用によって、言語表現にリズムや文体的奥行きが生まれる現象を「襷掛け効果」と呼ぶ。
}

\newcommand{\TasukiGakeEffectEn}{
\textbf{Tasuki-gake effect} refers to a stylistic phenomenon where Immediate Grammar and Adjustive Grammar are used in contexts that cross their typical domains of application.
When Immediate Grammar appears in a normally Adjustive context (such as a novel or formal speech), it injects liveliness and immediacy.
Conversely, when Adjustive Grammar is used in spontaneous utterances, it adds politeness or formality.
This interplay between grammatical modes creates a distinctive stylistic depth, referred to as the tasuki-gake effect.
}
