\ifJPN
\section{即時文法の事例集}
\else
\section{Examples of Immediate Grammar}
\fi
\label{sec:immediate_grammar_examples}

\ifJPN
\subsection{即時文法の実在性を支える観察の蓄積}
\else
\subsection{Accumulation of Observations Supporting the Existence of Immediate Grammar} 
\fi
\label{subsec:immediate_grammar_observations}

\ifJPN
本書では、「即時文法」という理論枠組みを提示しているが、その背景には、著者自身が行ってきた複数の実践的観察と記録がある。たとえば、『伊勢物語』や『土佐日記』の翻訳において、表現の自然な現代語訳を模索する過程で、従来の構文理論では説明が困難な即時的表現が多く見出された。また、aead(An expression a day)プロジェクトにおいては、日常的な日本語表現の中に即時文法の特徴を持つ表現を日々記録し、それらにタグと注釈を加えることで、言語使用の実態と即時文法の対応を浮かび上がらせてきた。
\else
In this book, we present the theoretical framework of "Immediate Grammar," which is supported by multiple practical observations and records made by the author. For example, in the process of translating works like "Ise Monogatari" and "Tosa Nikki," many immediate expressions were found that are difficult to explain using conventional syntactic theories. Additionally, in the aead (An expression a day) project, we have been recording expressions with characteristics of immediate grammar in everyday Japanese, tagging them, and adding annotations to highlight the correspondence between language use and immediate grammar.
\fi

\ifJPN
これらの作業は、直観的な主張ではなく、時間をかけて蓄積された観察の成果である。数百に及ぶ用例が、それぞれ即時文法の視点から解釈・記述されており、結果として、即時文法という理論が記述モデルとして有効であることを示す実証的な手がかりとなっている。
\else
These tasks are not based on intuitive claims but are the results of observations accumulated over time. Hundreds of examples have been interpreted and described from the perspective of immediate grammar, providing empirical clues that demonstrate the effectiveness of immediate grammar as a descriptive model.
\fi

%\subsection{土佐日記「よねさけ、しばしばくる」の語順と焦点化} \label{ex:yonosake}

%\subsection{伊勢物語「それがこうなる」の指示と文脈依存性} \label{ex:soregakounaru}

%\subsection{即時文法における「て連鎖」の認知処理} \label{ex:te-rensa}
